% Options for packages loaded elsewhere
% Options for packages loaded elsewhere
\PassOptionsToPackage{unicode}{hyperref}
\PassOptionsToPackage{hyphens}{url}
\PassOptionsToPackage{dvipsnames,svgnames,x11names}{xcolor}
%
\documentclass[
  croatian,
  11pt,
]{article}
\usepackage{xcolor}
\usepackage[margin=2.5cm]{geometry}
\usepackage{amsmath,amssymb}
\setcounter{secnumdepth}{-\maxdimen} % remove section numbering
\usepackage{iftex}
\ifPDFTeX
  \usepackage[T1]{fontenc}
  \usepackage[utf8]{inputenc}
  \usepackage{textcomp} % provide euro and other symbols
\else % if luatex or xetex
  \usepackage{unicode-math} % this also loads fontspec
  \defaultfontfeatures{Scale=MatchLowercase}
  \defaultfontfeatures[\rmfamily]{Ligatures=TeX,Scale=1}
\fi
\usepackage{lmodern}
\ifPDFTeX\else
  % xetex/luatex font selection
\fi
% Use upquote if available, for straight quotes in verbatim environments
\IfFileExists{upquote.sty}{\usepackage{upquote}}{}
\IfFileExists{microtype.sty}{% use microtype if available
  \usepackage[]{microtype}
  \UseMicrotypeSet[protrusion]{basicmath} % disable protrusion for tt fonts
}{}
\makeatletter
\@ifundefined{KOMAClassName}{% if non-KOMA class
  \IfFileExists{parskip.sty}{%
    \usepackage{parskip}
  }{% else
    \setlength{\parindent}{0pt}
    \setlength{\parskip}{6pt plus 2pt minus 1pt}}
}{% if KOMA class
  \KOMAoptions{parskip=half}}
\makeatother
% Make \paragraph and \subparagraph free-standing
\makeatletter
\ifx\paragraph\undefined\else
  \let\oldparagraph\paragraph
  \renewcommand{\paragraph}{
    \@ifstar
      \xxxParagraphStar
      \xxxParagraphNoStar
  }
  \newcommand{\xxxParagraphStar}[1]{\oldparagraph*{#1}\mbox{}}
  \newcommand{\xxxParagraphNoStar}[1]{\oldparagraph{#1}\mbox{}}
\fi
\ifx\subparagraph\undefined\else
  \let\oldsubparagraph\subparagraph
  \renewcommand{\subparagraph}{
    \@ifstar
      \xxxSubParagraphStar
      \xxxSubParagraphNoStar
  }
  \newcommand{\xxxSubParagraphStar}[1]{\oldsubparagraph*{#1}\mbox{}}
  \newcommand{\xxxSubParagraphNoStar}[1]{\oldsubparagraph{#1}\mbox{}}
\fi
\makeatother


\usepackage{longtable,booktabs,array}
\usepackage{calc} % for calculating minipage widths
% Correct order of tables after \paragraph or \subparagraph
\usepackage{etoolbox}
\makeatletter
\patchcmd\longtable{\par}{\if@noskipsec\mbox{}\fi\par}{}{}
\makeatother
% Allow footnotes in longtable head/foot
\IfFileExists{footnotehyper.sty}{\usepackage{footnotehyper}}{\usepackage{footnote}}
\makesavenoteenv{longtable}
\usepackage{graphicx}
\makeatletter
\newsavebox\pandoc@box
\newcommand*\pandocbounded[1]{% scales image to fit in text height/width
  \sbox\pandoc@box{#1}%
  \Gscale@div\@tempa{\textheight}{\dimexpr\ht\pandoc@box+\dp\pandoc@box\relax}%
  \Gscale@div\@tempb{\linewidth}{\wd\pandoc@box}%
  \ifdim\@tempb\p@<\@tempa\p@\let\@tempa\@tempb\fi% select the smaller of both
  \ifdim\@tempa\p@<\p@\scalebox{\@tempa}{\usebox\pandoc@box}%
  \else\usebox{\pandoc@box}%
  \fi%
}
% Set default figure placement to htbp
\def\fps@figure{htbp}
\makeatother



\ifLuaTeX
\usepackage[bidi=basic]{babel}
\else
\usepackage[bidi=default]{babel}
\fi
% get rid of language-specific shorthands (see #6817):
\let\LanguageShortHands\languageshorthands
\def\languageshorthands#1{}


\setlength{\emergencystretch}{3em} % prevent overfull lines

\providecommand{\tightlist}{%
  \setlength{\itemsep}{0pt}\setlength{\parskip}{0pt}}



 


\makeatletter
\@ifpackageloaded{caption}{}{\usepackage{caption}}
\AtBeginDocument{%
\ifdefined\contentsname
  \renewcommand*\contentsname{Table of contents}
\else
  \newcommand\contentsname{Table of contents}
\fi
\ifdefined\listfigurename
  \renewcommand*\listfigurename{List of Figures}
\else
  \newcommand\listfigurename{List of Figures}
\fi
\ifdefined\listtablename
  \renewcommand*\listtablename{List of Tables}
\else
  \newcommand\listtablename{List of Tables}
\fi
\ifdefined\figurename
  \renewcommand*\figurename{Figure}
\else
  \newcommand\figurename{Figure}
\fi
\ifdefined\tablename
  \renewcommand*\tablename{Table}
\else
  \newcommand\tablename{Table}
\fi
}
\@ifpackageloaded{float}{}{\usepackage{float}}
\floatstyle{ruled}
\@ifundefined{c@chapter}{\newfloat{codelisting}{h}{lop}}{\newfloat{codelisting}{h}{lop}[chapter]}
\floatname{codelisting}{Listing}
\newcommand*\listoflistings{\listof{codelisting}{List of Listings}}
\makeatother
\makeatletter
\makeatother
\makeatletter
\@ifpackageloaded{caption}{}{\usepackage{caption}}
\@ifpackageloaded{subcaption}{}{\usepackage{subcaption}}
\makeatother
\usepackage{bookmark}
\IfFileExists{xurl.sty}{\usepackage{xurl}}{} % add URL line breaks if available
\urlstyle{same}
\hypersetup{
  pdftitle={Računalna analiza teksta u istraživanju masovne komunikacije},
  pdflang={hr},
  colorlinks=true,
  linkcolor={blue},
  filecolor={Maroon},
  citecolor={Blue},
  urlcolor={Blue},
  pdfcreator={LaTeX via pandoc}}


\title{Računalna analiza teksta u istraživanju masovne komunikacije}
\author{}
\date{}
\begin{document}
\maketitle


\section{Uvod u računalnu analizu
teksta}\label{uvod-u-raux10dunalnu-analizu-teksta}

Zamislimo istraživača koji proučava medijsku konstrukciju klimatskih
promjena u hrvatskom javnom prostoru. Pred sobom ima korpus od stotinu
tisuća članaka objavljenih na vodećim informativnim portalima tijekom
posljednjeg desetljeća, tisuće političkih govora, stotine tisuća
komentara čitatelja i objava na društvenim mrežama. Tradicionalnim
metodama ručne analize sadržaja, čak i s timom od desetak kodirača koji
rade puno radno vrijeme, analiza ovako opsežnog materijala trajala bi
godinama. No pitanja koja istraživač želi postaviti zahtijevaju upravo
takav opseg jer se tiču suptilnih promjena u diskursu kroz vrijeme,
razlika između medija različitih političkih orijentacija, dinamike javne
rasprave i načina na koji se znanstveni konsenzus prevodi u medijske
narative. Upravo u takvim situacijama računalna analiza teksta postaje
nezamjenjiv metodološki alat.

Računalna analiza teksta, poznata i pod nazivima \emph{text mining},
\emph{computational text analysis} ili \emph{text as data}, predstavlja
skup metoda koje koriste računalne algoritme za sustavno ekstrahiranje
informacija iz nestrukturiranih tekstualnih podataka. U kontekstu
istraživanja masovne komunikacije, ove metode omogućuju analizu korpusa
veličine i složenosti koja bi bila nedostižna tradicionalnim pristupima,
otvarajući nova istraživačka pitanja i pružajući empirijsku osnovu za
teorije koje su ranije bile testirane samo na ograničenim uzorcima.

Važnost računalne analize teksta za komunikološka istraživanja proizlazi
iz nekoliko konvergentnih trendova. S jedne strane, digitalizacija
medijskog prostora generira enormne količine tekstualnih podataka koji
dokumentiraju javnu komunikaciju u neviđenoj granularnosti. Svaki članak
na informativnom portalu, svaka objava na društvenim mrežama, svaki
komentar čitatelja predstavlja potencijalni izvor uvida u dinamiku javne
sfere. S druge strane, napredak u računalnoj lingvistici i strojnom
učenju pruža sve sofisticiranije alate za obradu i analizu tih podataka.
Konvergencija obilnih podataka i moćnih alata otvara nove horizonte za
empirijsko istraživanje komunikacijskih fenomena.

Međutim, ulazak u svijet računalne analize teksta zahtijeva od
istraživača ne samo tehničke vještine, već i temeljito epistemološko
razumijevanje pretpostavki, mogućnosti i ograničenja ovih metoda. Četiri
važne pretpostavke koje vrijede za sve metode računalne analize teksta i
koje svaki istraživač mora internalizirati prije nego što pristupi
primjeni ovih tehnika.

Prva pretpostavka glasi da sve kvantitativne modele tekstualnih podataka
treba tretirati kao pogrešne, ali potencijalno korisne.Druga
pretpostavka ističe da kvantitativne metode za tekst pojačavaju ljudske
sposobnosti, ali ih ne zamjenjuju. Računalna analiza može obraditi
količine teksta koje bi bile nemoguće za ljudsku analizu, identificirati
obrasce koji bi promakli ljudskom oku i kvantificirati fenomene na način
koji omogućuje statističko zaključivanje. Međutim, računala ne
``razumiju'' tekst u smislu u kojem to čine ljudi. Interpretacija
rezultata, procjena njihove smislenosti, povezivanje s teorijskim
okvirima i donošenje zaključaka ostaju ljudske zadaće. Pokušaj potpune
automatizacije istraživačkog procesa gotovo sigurno vodi do trivialnih
ili pogrešnih zaključaka. Treća pretpostavka naglašava da je potrebna
ekstenzivna validacija specifična za problem. Ne postoji univerzalni
algoritam koji funkcionira za sve istraživačke zadatke i sve korpuse.
Metoda koja izvrsno funkcionira za klasifikaciju vijesti prema temama
možda će podbaciti za analizu sentimenta u komentarima na društvenim
mrežama. Algoritam treniran na engleskim tekstovima neće nužno dobro
funkcionirati na hrvatskim tekstovima. Zato je svako istraživanje dužno
uključiti rigoroznu validaciju koja demonstrira da odabrane metode
proizvode valjane rezultate u specifičnom kontekstu primjene. Četvrta
pretpostavka upozorava da računalne metode za tekst ne otklanjaju
potrebu za pomnim čitanjem i analitičkim razmišljanjem. Upravo suprotno,
one zahtijevaju još više pomnog čitanja jer istraživač mora razumjeti
podatke koje analizira, mora moći procijeniti smislenost rezultata
algoritama i mora znati interpretirati kvantitativne nalaze u svjetlu
kvalitativnog razumijevanja tekstualnog materijala. Računalna analiza ne
eliminira interpretaciju, već joj pruža kvantitativnu podlogu.

Razumijevanje ovih pretpostavki posebno je važno u kontekstu
komunikoloških istraživanja gdje su pitanja značenja, interpretacije i
konteksta centralna. Za razliku od analize numeričkih podataka gdje
postoji relativno jasna veza između mjerenja i koncepta, tekstualni
podaci zahtijevaju interpretativni korak koji model ne može obaviti
autonomno. Algoritam može prebrojati riječi, izračunati statističke
obrasce i grupirati dokumente prema sličnosti, ali ne može razumjeti
ironiju, prepoznati kulturne reference ili procijeniti relevantnost
nalaza za teorijska pitanja discipline.

Ukratko, računalna analiza teksta nije čarobno rješenje koje
automatizira istraživački proces, već sofisticirani alat koji, kada se
koristi promišljeno, može značajno proširiti doseg i dubinu
komunikoloških istraživanja. Uspješna primjena zahtijeva ne samo
tehničke vještine, već i teorijsku sofisticiranost, kritičku refleksiju
i kontinuirani dijalog između kvantitativnih metoda i kvalitativnog
razumijevanja.

Ovo poglavlje udžbenika vodi čitatelja kroz cijeli proces računalne
analize teksta, od pripreme sirovih podataka do interpretacije
rezultata. Započinjemo s postupcima pripreme podataka koji
transformiraju nestrukturirani tekst u oblik pogodan za analizu. Zatim
razmatramo različite načine reprezentacije teksta koji omogućuju njegovu
kvantitativnu obradu. Predstavljamo temeljne pristupe analizi,
uključujući nadzirano strojno učenje, tematsko modeliranje, analizu
sentimenta i ekstrakciju entiteta. Prelazimo na sofisticiranije tehnike
analize na razini diskursa, uključujući analizu kolokacija, okvira i
retoričke strukture. Kritički razmatramo ograničenja metoda i etička
pitanja koja postavljaju. Zaključujemo pogledom na buduće pravce razvoja
polja. Kroz cijeli tekst oslanjamo se na primjere iz hrvatskog medijskog
prostora i naglašavamo specifične izazove koji proizlaze iz
karakteristika hrvatskog jezika.

\section{Priprema podataka za
analizu}\label{priprema-podataka-za-analizu}

Zamislimo situaciju u kojoj istraživač masovne komunikacije želi
analizirati više od pedeset tisuća komentara čitatelja objavljenih ispod
članaka vodećih hrvatskih informativnih portala tijekom jedne izborne
kampanje. Sirovi tekst koji prikuplja iz digitalnog okruženja
predstavlja kaotičan niz znakova, interpunkcijskih oznaka, pogrešno
napisanih riječi, emotikona i raznovrsnih tipografskih varijacija.
Komentar poput ``neznam sta bi reko o ovim politicarima\ldots{} DOSTA
JE!!! 😡😡😡 \#izbori2024'' sadrži pravopisne pogreške, ispuštene
dijakritičke znakove, višestruke interpunkcijske znakove, emotikone i
hashtag. Takav materijal u svom izvornom obliku nije pogodan za sustavnu
analizu jer analitičke metode zahtijevaju određenu razinu
strukturiranosti i konzistentnosti podataka. Priprema podataka stoga
predstavlja temeljni korak koji prethodi svakoj računalno potpomognutoj
analizi teksta, a njezina kvaliteta izravno određuje valjanost i
pouzdanost konačnih rezultata istraživanja.

Proces pripreme podataka obuhvaća niz postupaka kojima se
nestrukturirani tekst transformira u oblik prikladan za kvantitativnu
obradu. Može se reći da ovaj postupak predstavlja svojevrsni most između
sirovog jezičnog materijala i njegova numeričkog prikaza koji omogućuje
primjenu statističkih i računalnih metoda. Valja napomenuti da odluke
donesene u ovoj fazi imaju dalekosežne posljedice jer svako
pojednostavljenje teksta nužno uključuje određeni gubitak informacija.
Istraživač stoga mora pažljivo balansirati između potrebe za redukcijom
složenosti i očuvanja semantički relevantnih svojstava teksta.

Priprema podataka nije neutralan tehnički postupak već epistemološki čin
koji odražava teorijske pretpostavke istraživača o prirodi jezika i
komunikacije. Kada odlučujemo koje elemente teksta zadržati, a koje
odbaciti, implicitno definiramo što smatramo značajnim za naše
istraživačko pitanje. Upravo zato je od iznimne važnosti da istraživač
razumije logiku svakoga koraka u procesu pripreme te da svoje odluke
može argumentirano obrazložiti.

\subsection{Tokenizacija kao temeljna
operacija}\label{tokenizacija-kao-temeljna-operacija}

Prva i najtemeljnija operacija u pripremi tekstualnih podataka jest
tokenizacija, postupak raščlanjivanja kontinuiranog niza teksta na
diskretne jedinice koje nazivamo tokenima. Token predstavlja najmanju
jedinicu analize, a ovisno o istraživačkom pitanju može predstavljati
pojedinačnu riječ, n-gram, rečenicu ili čak odlomak. U kontekstu
istraživanja masovne komunikacije najčešće se kao token koristi
pojedinačna riječ budući da riječi predstavljaju temeljne nositelje
značenja u jeziku.

Premda se tokenizacija može činiti trivijalnom operacijom koja se svodi
na razdvajanje teksta prema razmacima, u praksi ovaj postupak uključuje
niz odluka koje mogu značajno utjecati na rezultate analize. Promotrimo
rečenicu ``Predsjednik Vlade RH g. Plenković posjetio je tvrtku INA-u.''
Jednostavna tokenizacija prema razmacima proizvela bi tokene:
{[}``Predsjednik'', ``Vlade'', ``RH'', ``g.'', ``Plenković'',
``posjetio'', ``je'', ``tvrtku'', ``INA-u.''{]}. Međutim, ovakva
tokenizacija postavlja nekoliko pitanja. Treba li ``RH'' tretirati kao
token ili proširiti u ``Republika Hrvatska''? Kako postupiti s kraticom
``g.'' koja uključuje točku? Je li ``INA-u.'' jedan token ili treba
odvojiti interpunkciju? Kako postupiti s crticom unutar naziva? Odgovori
na ova pitanja ovise o specifičnostima istraživačkog pitanja i
karakteristikama korpusa.

Tablica 2 prikazuje različite strategije tokenizacije i njihove
rezultate za odabrani primjer rečenice.

\begin{longtable}[]{@{}
  >{\raggedright\arraybackslash}p{(\linewidth - 6\tabcolsep) * \real{0.3205}}
  >{\raggedright\arraybackslash}p{(\linewidth - 6\tabcolsep) * \real{0.2821}}
  >{\raggedright\arraybackslash}p{(\linewidth - 6\tabcolsep) * \real{0.2308}}
  >{\raggedright\arraybackslash}p{(\linewidth - 6\tabcolsep) * \real{0.1667}}@{}}
\caption{Usporedba strategija tokenizacije na primjeru rečenice
``Predsjednik Vlade RH g. Plenković posjetio je tvrtku
INA-u.''}\tabularnewline
\toprule\noalign{}
\begin{minipage}[b]{\linewidth}\raggedright
Strategija tokenizacije
\end{minipage} & \begin{minipage}[b]{\linewidth}\raggedright
Rezultat za ``INA-u.''
\end{minipage} & \begin{minipage}[b]{\linewidth}\raggedright
Rezultat za ``g.''
\end{minipage} & \begin{minipage}[b]{\linewidth}\raggedright
Broj tokena
\end{minipage} \\
\midrule\noalign{}
\endfirsthead
\toprule\noalign{}
\begin{minipage}[b]{\linewidth}\raggedright
Strategija tokenizacije
\end{minipage} & \begin{minipage}[b]{\linewidth}\raggedright
Rezultat za ``INA-u.''
\end{minipage} & \begin{minipage}[b]{\linewidth}\raggedright
Rezultat za ``g.''
\end{minipage} & \begin{minipage}[b]{\linewidth}\raggedright
Broj tokena
\end{minipage} \\
\midrule\noalign{}
\endhead
\bottomrule\noalign{}
\endlastfoot
Jednostavna (razmaci) & INA-u. & g. & 9 \\
S odvajanjem interpunkcije & INA-u, . & g, . & 11 \\
S rastavljanjem crtica & INA, -, u, . & g, . & 13 \\
Normalizirana & ina & g & 8 \\
\end{longtable}

Za hrvatske tekstove dodatnu komplikaciju predstavljaju enklitike,
kratke nenaglašene riječi koje se u govoru vežu uz prethodnu ili
sljedeću riječ. U pisanom tekstu one su odvojene razmacima, no njihova
gramatička funkcija vezana je uz druge riječi u rečenici. Primjerice, u
rečenici ``Vidjeli smo ga'' zamjenica ``ga'' funkcionira kao objekt
glagola, no tokenizacija je tretira kao zasebnu jedinicu, što je
ispravno za većinu analiza.

\subsection{Uklanjanje zaustavnih
riječi}\label{uklanjanje-zaustavnih-rijeux10di}

Nakon tokenizacije, korpus tipično sadrži velik broj visokofrekventnih
riječi koje nose malo semantičke vrijednosti. Riječi poput ``i'',
``ili'', ``je'', ``su'', ``na'', ``u'', ``za'' pojavljuju se u gotovo
svakom dokumentu i dominiraju distribucijom frekvencija. Ove se riječi
konvencionalno nazivaju zaustavnim riječima jer ne doprinose
razlikovanju dokumenata prema sadržaju. U analizi tematske strukture
korpusa novinarskih članaka, činjenica da svi članci sadrže veznik ``i''
ne govori nam ništa o njihovu sadržaju. Uklanjanje zaustavnih riječi
stogaa smanjuje dimenzionalnost podataka i poboljšava omjer signala i
šuma u analizi. Tipična lista zaustavnih riječi za hrvatski jezik sadrži
nekoliko stotina najčešćih riječi uključujući članove, veznike,
prijedloge, pomoćne glagole i najčešće zamjenice. Međutim, sastavljanje
takve liste nije trivijalan zadatak i uključuje interpretativne odluke.
Je li riječ ``može'' zaustavna riječ ili sadržajna riječ? Odgovor ovisi
o kontekstu. U korpusu političkih govora, modalni glagoli poput
``može'', ``mora'', ``treba'' mogu nositi značajnu informaciju o
diskurzivnim strategijama govornika. U tehničkim tekstovima, iste riječi
možda nemaju takvu funkciju.

Tablica 3 prikazuje primjer tipičnih hrvatskih zaustavnih riječi
organiziranih prema vrstama riječi.

\begin{longtable}[]{@{}
  >{\raggedright\arraybackslash}p{(\linewidth - 2\tabcolsep) * \real{0.3333}}
  >{\raggedright\arraybackslash}p{(\linewidth - 2\tabcolsep) * \real{0.6667}}@{}}
\caption{Primjeri hrvatskih zaustavnih riječi prema vrstama
riječi}\tabularnewline
\toprule\noalign{}
\begin{minipage}[b]{\linewidth}\raggedright
Vrsta riječi
\end{minipage} & \begin{minipage}[b]{\linewidth}\raggedright
Primjeri zaustavnih riječi
\end{minipage} \\
\midrule\noalign{}
\endfirsthead
\toprule\noalign{}
\begin{minipage}[b]{\linewidth}\raggedright
Vrsta riječi
\end{minipage} & \begin{minipage}[b]{\linewidth}\raggedright
Primjeri zaustavnih riječi
\end{minipage} \\
\midrule\noalign{}
\endhead
\bottomrule\noalign{}
\endlastfoot
Veznici & i, ili, ali, nego, jer, da, kako, dok, kad \\
Prijedlozi & u, na, za, s, sa, od, do, iz, po, o, prema \\
Zamjenice & ja, ti, on, ona, ono, mi, vi, oni, one, ona, ovaj, taj,
onaj \\
Pomoćni glagoli & sam, si, je, smo, ste, su, biti, jesam, nisam \\
Čestice & li, ne, ni, još, već, samo, baš, čak \\
\end{longtable}

Uklanjanje zaustavnih riječi može se provesti korištenjem unaprijed
definiranih lista ili automatski, identificiranjem riječi čija
frekvencija prelazi određeni prag. Oba pristupa imaju prednosti i
nedostatke. Unaprijed definirane liste omogućuju preciznu kontrolu, ali
zahtijevaju prilagodbu specifičnom korpusu. Automatski pristupi su
fleksibilniji, ali mogu ukloniti riječi koje su relevantne za specifično
istraživačko pitanje. Preporučljivo je da istraživač pregleda listu
zaustavnih riječi i prilagodi je potrebama svog istraživanja.

\subsection{Stemizacija i
lematizacija}\label{stemizacija-i-lematizacija}

Infleskivna priroda hrvatskog jezika rezultira time da se ista riječ
pojavljuje u mnogim različitim oblicima ovisno o gramatičkom kontekstu.
Glagol ``govoriti'' može se pojaviti kao ``govorim'', ``govoriš'',
``govori'', ``govorimo'', ``govorite'', ``govore'', ``govorio'',
``govorila'', ``govorili'', ``govorile'', ``govoreći'' i u brojnim
drugim oblicima. Za analitičke svrhe, sve ove oblike želimo tretirati
kao različite varijante iste temeljne konceptualne jedinice. Stemizacija
i lematizacija dva su komplementarna pristupa rješavanju ovog problema,
pri čemu oba teže svođenju različitih morfoloških varijanti riječi na
zajednički korijen. Njihova temeljna logika je intuitivno jasna jer ako
znamo da oblici ``ekonomija'', ``ekonomije'', ``ekonomski'' i
``ekonomista'' dijele zajedničku semantičku jezgru, ima smisla tretirati
ih kao iste konceptualne jedinice.

Stemizacija predstavlja heuristički pristup koji koristi skup pravila za
uklanjanje sufiksa i prefiksa s riječi kako bi se dobio korijen ili
stem. Algoritmi za stemizaciju operiraju isključivo nad oblikom riječi
bez obzira na njezino značenje ili gramatičku funkciju u rečenici.
Rezultat stemizacije nije nužno valjana riječ u jeziku već apstraktni
korijen koji služi kao oznaka klase morfološki srodnih riječi.
Primjerice, stemizacija riječi ``ekonomski'', ``ekonomije'' i
``ekonomista'' mogla bi proizvesti korijen ``ekonom'' koji nije
samostalna hrvatska riječ, ali služi kao zajednički identifikator za sve
tri izvorne riječi. Lematizacija, s druge strane, predstavlja
lingvistički sofisticiraniji pristup koji svodi riječi na njihov
kanonski oblik ili lemu, odnosno oblik koji bi se pronašao kao natuknica
u rječniku. Za imenice to je nominativ jednine, za glagole infinitiv, za
pridjeve nominativ jednine muškog roda. Za razliku od stemizacije,
lematizacija uzima u obzir kontekst u kojem se riječ javlja i njezinu
gramatičku funkciju. Tako će riječ ``bolje'' biti ispravno svedena na
lemu ``dobar'' kao komparativ pridjeva, dok bi stemizacija vjerojatno
proizvela nevaljani korijen poput ``bolj''.

Tablica 4 ilustrira razliku između stemizacije i lematizacije na
konkretnim primjerima.

\begin{longtable}[]{@{}
  >{\raggedright\arraybackslash}p{(\linewidth - 6\tabcolsep) * \real{0.2143}}
  >{\raggedright\arraybackslash}p{(\linewidth - 6\tabcolsep) * \real{0.3143}}
  >{\raggedright\arraybackslash}p{(\linewidth - 6\tabcolsep) * \real{0.3286}}
  >{\raggedright\arraybackslash}p{(\linewidth - 6\tabcolsep) * \real{0.1429}}@{}}
\caption{Usporedba stemizacije i lematizacije za odabrane
riječi}\tabularnewline
\toprule\noalign{}
\begin{minipage}[b]{\linewidth}\raggedright
Izvorni oblik
\end{minipage} & \begin{minipage}[b]{\linewidth}\raggedright
Rezultat stemizacije
\end{minipage} & \begin{minipage}[b]{\linewidth}\raggedright
Rezultat lematizacije
\end{minipage} & \begin{minipage}[b]{\linewidth}\raggedright
Napomena
\end{minipage} \\
\midrule\noalign{}
\endfirsthead
\toprule\noalign{}
\begin{minipage}[b]{\linewidth}\raggedright
Izvorni oblik
\end{minipage} & \begin{minipage}[b]{\linewidth}\raggedright
Rezultat stemizacije
\end{minipage} & \begin{minipage}[b]{\linewidth}\raggedright
Rezultat lematizacije
\end{minipage} & \begin{minipage}[b]{\linewidth}\raggedright
Napomena
\end{minipage} \\
\midrule\noalign{}
\endhead
\bottomrule\noalign{}
\endlastfoot
govorila & govor & govoriti & Glagol u prošlom vremenu \\
gospodarstva & gospodar & gospodarstvo & Imenica u genitivu \\
europskim & europs & europski & Pridjev u instrumentalu \\
bolje & bolj & dobar & Komparativ pridjeva \\
novinarka & novinar & novinarka & Imenica ženskog roda \\
\end{longtable}

Odluka između stemizacije i lematizacije ovisi o specifičnostima
istraživačkog pitanja i dostupnim resursima. Stemizacija je računalno
učinkovitija i ne zahtijeva opsežne lingvističke resurse, što je čini
praktičnom za brzu obradu velikih korpusa. Međutim, njezina agresivnost
može rezultirati gubitkom semantičkih distinkcija ili, suprotno,
neuspjehom u prepoznavanju morfološki nepravilnih oblika. Lematizacija
pruža veću preciznost i zadržava semantičku koherentnost, ali zahtijeva
sofisticirane lingvističke alate uključujući leksičke baze podataka i
sustave za označavanje vrsta riječi.

\subsection{Izazovi specifični za hrvatski
jezik}\label{izazovi-specifiux10dni-za-hrvatski-jezik}

Primjena metoda računalne analize teksta na hrvatski jezik suočava se s
nizom specifičnih izazova koji proizlaze iz lingvističkih karakteristika
hrvatskog i relativne oskudnosti dostupnih jezičnih resursa. Dok su
alati i metode razvijene pretežno za engleski jezik dostigli visoku
razinu sofisticiranosti, njihova prilagodba morfološki bogatijim i
resursno siromašnijim jezicima poput hrvatskog ostaje aktivan
istraživački problem. Morfološka složenost hrvatskog jezika stoga
predstavlja izazov računalnoj obradi teksta. Hrvatski pripada skupini
slavenskih jezika s bogatom flektivnom morfologijom, što znači da
imenice, pridjevi i zamjenice poznaju sedam padeža u jednini i množini,
dok glagoli variraju prema licu, broju, vremenu, načinu i vidu.
Posljedica toga je izrazito visok broj različitih oblika koje ista
leksička jedinica može poprimiti. Procjenjuje se da prosječna hrvatska
imenica ima četrnaest različitih morfoloških realizacija, dok glagoli
mogu imati i preko stotinu oblika kada se uključe svi aspekti
konjugacije. Ova morfološka raznolikost drastično povećava
dimenzionalnost podataka i otežava prepoznavanje obrazaca.

Relativno slobodan redoslijed riječi u hrvatskom predstavlja dodatnu
komplikaciju za metode koje se oslanjaju na sekvencijalne obrasce poput
analize n-grama. Dok u engleskom jeziku pozicija riječi unutar rečenice
ima snažnu gramatičku funkciju, u hrvatskom se ista informacija kodira
morfološkim nastavcima, a redoslijed riječi služi pretežno pragmatičkim
i stilskim funkcijama. Rečenice ``Marko je vidio Anu'', ``Anu je vidio
Marko'' i ``Vidio je Marko Anu'' izražavaju istu propoziciju, ali s
različitim informacijskim fokusom. To znači da ista propozicija može
biti izražena na više sintaktički različitih načina, što komplicira
usporedbu tekstova i identifikaciju obrazaca.

Nedostatak lingvističkih resursa za hrvatski jezik predstavlja praktičnu
prepreku implementaciji sofisticiranih metoda analize. Dok za engleski
jezik postoje opsežne leksičke baze poput WordNeta s desecima tisuća
sinkroniziranih koncepata, validirani rječnici sentimenata s desecima
tisuća označenih riječi te napredni sustavi za morfološku analizu, za
hrvatski su takvi resursi znatno oskudniji. Hrvatski WordNet postoji,
ali je manjeg opsega od engleskog originala. Rječnici sentimenata za
hrvatski razvijaju se u akademskim projektima poput CroSentiLex, no
njihova pokrivenost i validacija variraju.

Specifičnosti digitalnog registra hrvatskog jezika dodatno usložnjavaju
analizu tekstova s društvenih mreža i komentara na portalima. Korisnici
često koriste nestandardne oblike pisanja uključujući ispuštanje
dijakritičkih znakova, uporabu engleskih riječi i fraza, regionalizme i
žargonizme te razne oblike kreativnog pravopisa. Tekst poput ``neznam
sta bi reko, bas mi je bed'' uključuje više odstupanja od standardnog
jezika koja algoritmi pripremljeni za standardni hrvatski neće ispravno
obraditi.

Tablica 5 sažima ključne izazove hrvatskog jezika za računalnu analizu
teksta.

\begin{longtable}[]{@{}
  >{\raggedright\arraybackslash}p{(\linewidth - 4\tabcolsep) * \real{0.1702}}
  >{\raggedright\arraybackslash}p{(\linewidth - 4\tabcolsep) * \real{0.3191}}
  >{\raggedright\arraybackslash}p{(\linewidth - 4\tabcolsep) * \real{0.5106}}@{}}
\caption{Izazovi primjene računalne analize teksta na hrvatski
jezik}\tabularnewline
\toprule\noalign{}
\begin{minipage}[b]{\linewidth}\raggedright
Izazov
\end{minipage} & \begin{minipage}[b]{\linewidth}\raggedright
Opis problema
\end{minipage} & \begin{minipage}[b]{\linewidth}\raggedright
Implikacije za analizu
\end{minipage} \\
\midrule\noalign{}
\endfirsthead
\toprule\noalign{}
\begin{minipage}[b]{\linewidth}\raggedright
Izazov
\end{minipage} & \begin{minipage}[b]{\linewidth}\raggedright
Opis problema
\end{minipage} & \begin{minipage}[b]{\linewidth}\raggedright
Implikacije za analizu
\end{minipage} \\
\midrule\noalign{}
\endhead
\bottomrule\noalign{}
\endlastfoot
Morfološka složenost & 7 padeža, bogata konjugacija & Visoka
dimenzionalnost, otežano grupiranje \\
Slobodan redoslijed riječi & Gramatika kodirana morfemima & Ograničena
korisnost sekvencijalnih metoda \\
Oskudica resursa & Manji leksikoni i alati & Niža preciznost jezičnih
alata \\
Digitalni registar & Ispuštanje dijakritika, anglizmi & Potrebna dodatna
normalizacija \\
\end{longtable}

\subsection{Čišćenje i normalizacija
teksta}\label{ux10diux161ux107enje-i-normalizacija-teksta}

Sirovi tekst prikupljen iz digitalnih izvora gotovo uvijek sadrži
elemente koji nisu relevantni za sadržajnu analizu, a mogu ometati rad
analitičkih alata ili iskriviti rezultate. Čišćenje teksta obuhvaća skup
postupaka kojima se uklanjaju takvi neželjeni elementi i standardizira
format podataka. Premda se može činiti tehničkim i rutinskim, ovaj korak
zahtijeva pažljivo razmatranje jer svaka odluka o uklanjanju ili
transformaciji utječe na konačnu analizu.

Uklanjanje interpunkcije jedan je od najčešćih koraka čišćenja. Točke,
zarezi, upitnici i drugi interpunkcijski znakovi obično ne nose
semantičku informaciju relevantnu za analizu sadržaja te se rutinski
uklanjaju. Međutim, postoje konteksti u kojima interpunkcija može biti
značajna. U analizi emocionalnog intenziteta višestruki uskličnici ili
upitnici mogu signalizirati pojačanu emocionalnu angažiranost autora, pa
bi njihovo uklanjanje značilo gubitak relevantne informacije.

Tekstovi prikupljeni s interneta često sadrže HTML oznake, URL adrese i
posebne znakove koji su relevantni za prikaz teksta u pregledniku, ali
nemaju sadržajnu vrijednost. Uklanjanje takvih elemenata dio je
standardnog protokola čišćenja. Posebnu pažnju zahtijevaju emotikoni i
emojiji koji su sve prisutniji u tekstovima digitalnih medija. Iako
nemaju leksičko značenje u tradicionalnom smislu, emojiji često nose
značajnu emocionalnu informaciju i mogu biti relevantni za određene
tipove analize poput analize sentimenata.

Normalizacija dijakritika specifičan je izazov za hrvatski jezik.
Korisnici digitalnih medija često ispuštaju dijakritičke znakove pišući
``zasto'' umjesto ``zašto'' ili ``covjek'' umjesto ``čovjek''.
Istraživač može odlučiti normalizirati takve oblike na standardnu
ortografiju ili pak zadržati nestandardne oblike kao indikator registra
ili sociolingvističkih karakteristika autora. Prva opcija
pojednostavljuje analizu dok druga zadržava potencijalno relevantnu
informaciju.

Tablica 6 prikazuje pregled koraka pripreme podataka i njihovih
implikacija za analizu.

\begin{longtable}[]{@{}
  >{\raggedright\arraybackslash}p{(\linewidth - 4\tabcolsep) * \real{0.1458}}
  >{\raggedright\arraybackslash}p{(\linewidth - 4\tabcolsep) * \real{0.1458}}
  >{\raggedright\arraybackslash}p{(\linewidth - 4\tabcolsep) * \real{0.7083}}@{}}
\caption{Pregled koraka pripreme podataka i njihovih implikacija za
analizu}\tabularnewline
\toprule\noalign{}
\begin{minipage}[b]{\linewidth}\raggedright
Korak
\end{minipage} & \begin{minipage}[b]{\linewidth}\raggedright
Svrha
\end{minipage} & \begin{minipage}[b]{\linewidth}\raggedright
Potencijalni gubitak informacija
\end{minipage} \\
\midrule\noalign{}
\endfirsthead
\toprule\noalign{}
\begin{minipage}[b]{\linewidth}\raggedright
Korak
\end{minipage} & \begin{minipage}[b]{\linewidth}\raggedright
Svrha
\end{minipage} & \begin{minipage}[b]{\linewidth}\raggedright
Potencijalni gubitak informacija
\end{minipage} \\
\midrule\noalign{}
\endhead
\bottomrule\noalign{}
\endlastfoot
Tokenizacija & Raščlamba teksta na analitičke jedinice & Gubitak
informacija o sekvencijalnim odnosima \\
Uklanjanje stop-riječi & Redukcija šuma i dimenzionalnosti & Gubitak
gramatičkih i pragmatičkih signala \\
Lematizacija/stemizacija & Svođenje na korijenske oblike & Gubitak
morfoloških distinkcija \\
Čišćenje i normalizacija & Standardizacija formata & Gubitak
stilističkih i registarskih signala \\
\end{longtable}

Uzevši sve navedeno u obzir, postupak pripreme podataka može se
konceptualizirati kao serija transformacija koje sirovi tekst prevode u
strukturirani oblik spreman za analizu. Svaki korak uključuje implicitne
i eksplicitne odluke o tome što je relevantno za istraživanje, a što
predstavlja šum koji treba ukloniti. Kvaliteta pripreme podataka izravno
utječe na valjanost rezultata, a transparentnost u dokumentiranju
primijenjenih postupaka nužan je preduvjet znanstvene ponovljivosti.
Istraživač koji razumije logiku i implikacije svakoga koraka u procesu
pripreme može donijeti informirane odluke koje optimalno služe
specifičnom istraživačkom pitanju.

\section{Reprezentacija teksta}\label{reprezentacija-teksta}

Zamislimo istraživača koji želi usporediti kako tri vodeća hrvatska
informativna portala izvještavaju o klimatskim promjenama. Na
raspolaganju mu je korpus od nekoliko tisuća članaka prikupljenih
tijekom jednogodišnjeg razdoblja. Nakon što je proveo tokenizaciju,
uklonio stop-riječi i normalizirao tekst, suočava se s temeljnim
pitanjem: kako pretvoriti ove tekstualne podatke u oblik koji će
omogućiti sustavnu usporedbu i statističku analizu? Kako reprezentirati
tekst na način koji će računalu omogućiti prepoznavanje sličnosti i
razlika između dokumenata, identificiranje karakterističnih tema za
svaki portal te otkrivanje obrazaca u medijskom diskursu?

Reprezentacija teksta odnosi se na postupke kojima se tekstualni podaci
transformiraju u matematičke strukture, najčešće vektore ili matrice,
pogodne za računalnu obradu. Ova transformacija predstavlja ključan
korak jer kvaliteta reprezentacije izravno utječe na uspješnost svih
naknadnih analitičkih postupaka. Loše odabrana reprezentacija može
prikriti relevantne obrasce ili, suprotno, proizvesti artefakte koji ne
odražavaju stvarne karakteristike teksta.

Temeljni izazov reprezentacije teksta proizlazi iz fundamentalne razlike
između prirode jezika i zahtjeva kvantitativnih metoda. Jezik je
semantički bogat, kontekstualno ovisan i inherentno višeznačan sustav u
kojem značenje proizlazi iz složenih odnosa između riječi, rečenica i
šireg diskursnog konteksta. S druge strane, statistički algoritmi
zahtijevaju precizno definirane numeričke vrijednosti organizirane u
pravilne strukture. Svaka reprezentacija stoga nužno uključuje određenu
razinu pojednostavljenja i gubitka informacija, a istraživačev zadatak
je odabrati pristup koji optimalno balansira između računalne
učinkovitosti i očuvanja semantički relevantnih svojstava teksta.

\subsection{Model vreće riječi}\label{model-vreux107e-rijeux10di}

Model vreće riječi, poznat i pod engleskim nazivom Bag-of-Words ili
skraćenicom BoW, predstavlja najjednostavniji i povijesno najraniji
pristup reprezentaciji teksta za kvantitativnu analizu. Osnovna
intuicija ovog modela proizlazi iz pretpostavke da se sadržaj dokumenta
može aproksimirati jednostavnim prebrojavanjem riječi koje se u njemu
pojavljuju, pri čemu se potpuno zanemaruje redoslijed riječi i
gramatička struktura. Metaforički rečeno, zamišljamo da sve riječi iz
dokumenta ubacujemo u veliku vreću, miješamo ih i potom samo brojimo
koliko puta se svaka riječ pojavljuje.

Premda se ova pretpostavka može činiti drastičnim pojednostavljenjem,
praksa je pokazala da za mnoge analitičke zadatke takva reprezentacija
pruža iznenađujuće dobre rezultate. Ako želimo klasificirati novinarske
članke prema temama, činjenica da članak sadrži riječi poput
``inflacija'', ``kamatna stopa'', ``BDP'' i ``proračun'' snažno sugerira
da se radi o ekonomskoj tematici, neovisno o tome kojim redoslijedom se
te riječi pojavljuju u tekstu. Slično tome, visoka frekvencija riječi
poput ``utakmica'', ``gol'', ``prvak'' i ``reprezentacija'' pouzdano
identificira sportski sadržaj.

Formalno, model vreće riječi reprezentira korpus dokumenata pomoću
matrice dokument-termin. Radi se o matrici u kojoj svaki redak
predstavlja jedan dokument, svaki stupac predstavlja jednu jedinstvenu
riječ iz cjelokupnog vokabulara korpusa, a vrijednost u svakoj ćeliji
označava frekvenciju pojavljivanja te riječi u tom dokumentu. Ako imamo
korpus od \(m\) dokumenata i vokabular od \(n\) jedinstvenih riječi,
rezultirajuća matrica ima dimenzije \(m \times n\).

Zamislimo konkretan primjer s tri kratka naslova novinskih članaka:
``Vlada najavljuje nove porezne reforme'', ``Premijer najavljuje reforme
zdravstvenog sustava'' i ``Nove mjere za poticanje gospodarstva''.
Matrica dokument-termin za ovaj mini-korpus prikazana je u tablici 7.

\begin{longtable}[]{@{}
  >{\raggedright\arraybackslash}p{(\linewidth - 22\tabcolsep) * \real{0.0847}}
  >{\centering\arraybackslash}p{(\linewidth - 22\tabcolsep) * \real{0.0593}}
  >{\centering\arraybackslash}p{(\linewidth - 22\tabcolsep) * \real{0.1017}}
  >{\centering\arraybackslash}p{(\linewidth - 22\tabcolsep) * \real{0.0508}}
  >{\centering\arraybackslash}p{(\linewidth - 22\tabcolsep) * \real{0.0763}}
  >{\centering\arraybackslash}p{(\linewidth - 22\tabcolsep) * \real{0.0763}}
  >{\centering\arraybackslash}p{(\linewidth - 22\tabcolsep) * \real{0.0847}}
  >{\centering\arraybackslash}p{(\linewidth - 22\tabcolsep) * \real{0.1186}}
  >{\centering\arraybackslash}p{(\linewidth - 22\tabcolsep) * \real{0.0763}}
  >{\centering\arraybackslash}p{(\linewidth - 22\tabcolsep) * \real{0.0593}}
  >{\centering\arraybackslash}p{(\linewidth - 22\tabcolsep) * \real{0.0932}}
  >{\centering\arraybackslash}p{(\linewidth - 22\tabcolsep) * \real{0.1186}}@{}}
\caption{Primjer matrice dokument-termin za tri kratka
dokumenta}\tabularnewline
\toprule\noalign{}
\begin{minipage}[b]{\linewidth}\raggedright
Dokument
\end{minipage} & \begin{minipage}[b]{\linewidth}\centering
vlada
\end{minipage} & \begin{minipage}[b]{\linewidth}\centering
najavljuje
\end{minipage} & \begin{minipage}[b]{\linewidth}\centering
nove
\end{minipage} & \begin{minipage}[b]{\linewidth}\centering
porezne
\end{minipage} & \begin{minipage}[b]{\linewidth}\centering
reforme
\end{minipage} & \begin{minipage}[b]{\linewidth}\centering
premijer
\end{minipage} & \begin{minipage}[b]{\linewidth}\centering
zdravstvenog
\end{minipage} & \begin{minipage}[b]{\linewidth}\centering
sustava
\end{minipage} & \begin{minipage}[b]{\linewidth}\centering
mjere
\end{minipage} & \begin{minipage}[b]{\linewidth}\centering
poticanje
\end{minipage} & \begin{minipage}[b]{\linewidth}\centering
gospodarstva
\end{minipage} \\
\midrule\noalign{}
\endfirsthead
\toprule\noalign{}
\begin{minipage}[b]{\linewidth}\raggedright
Dokument
\end{minipage} & \begin{minipage}[b]{\linewidth}\centering
vlada
\end{minipage} & \begin{minipage}[b]{\linewidth}\centering
najavljuje
\end{minipage} & \begin{minipage}[b]{\linewidth}\centering
nove
\end{minipage} & \begin{minipage}[b]{\linewidth}\centering
porezne
\end{minipage} & \begin{minipage}[b]{\linewidth}\centering
reforme
\end{minipage} & \begin{minipage}[b]{\linewidth}\centering
premijer
\end{minipage} & \begin{minipage}[b]{\linewidth}\centering
zdravstvenog
\end{minipage} & \begin{minipage}[b]{\linewidth}\centering
sustava
\end{minipage} & \begin{minipage}[b]{\linewidth}\centering
mjere
\end{minipage} & \begin{minipage}[b]{\linewidth}\centering
poticanje
\end{minipage} & \begin{minipage}[b]{\linewidth}\centering
gospodarstva
\end{minipage} \\
\midrule\noalign{}
\endhead
\bottomrule\noalign{}
\endlastfoot
D1 & 1 & 1 & 1 & 1 & 1 & 0 & 0 & 0 & 0 & 0 & 0 \\
D2 & 0 & 1 & 0 & 0 & 1 & 1 & 1 & 1 & 0 & 0 & 0 \\
D3 & 0 & 0 & 1 & 0 & 0 & 0 & 0 & 0 & 1 & 1 & 1 \\
\end{longtable}

Iz ove matrice možemo vidjeti da dokumenti D1 i D2 dijele dvije riječi
(``najavljuje'' i ``reforme''), dok D3 nema zajedničkih riječi s ostala
dva dokumenta osim pridjeva ``nove'' koji dijeli s D1. Ova informacija
može poslužiti za kvantifikaciju sličnosti između dokumenata, primjerice
korištenjem kosinusne sličnosti ili drugih mjera udaljenosti u
vektorskom prostoru.

Model vreće riječi ima nekoliko značajnih ograničenja koje treba imati
na umu. Budući da se potpuno zanemaruje redoslijed riječi, rečenice
``Marko voli Anu'' i ``Ana voli Marka'' imaju identičnu reprezentaciju
premda izražavaju različite propozicije. Slično tome, rečenica ``Film
nije bio dosadan'' i ``Film je bio dosadan'' imaju gotovo identičnu
reprezentaciju jer se razlikuju samo u jednoj riječi, premda izražavaju
suprotne evaluacije. Ova ograničenja motiviraju razvoj sofisticiranijih
metoda reprezentacije.

\subsection{TF-IDF: Vaganje važnosti
riječi}\label{tf-idf-vaganje-vaux17enosti-rijeux10di}

Jednostavno prebrojavanje riječi, kako ga provodi model vreće riječi,
tretira sve riječi kao jednako važne. Međutim, intuitivno je jasno da
sve riječi ne nose jednaku količinu informacija. Riječ koja se
pojavljuje u gotovo svakom dokumentu korpusa, poput veznika ``i'' ili
glagola ``je'', govori nam vrlo malo o sadržaju specifičnog dokumenta. S
druge strane, riječ koja se pojavljuje samo u malom broju dokumenata
može biti ključna za razumijevanje njihove specifičnosti. Primjerice,
riječ ``fotonaponski'' u korpusu ekonomskih vijesti vjerojatno se
pojavljuje samo u člancima o obnovljivim izvorima energije i kao takva
je visoko informativna za identificiranje te podteme.

TF-IDF, kratica za term frequency-inverse document frequency, jest
statistička mjera koja pokušava uhvatiti ovu intuiciju dodjeljivanjem
veće težine riječima koje su karakteristične za pojedine dokumente, a
manje težine riječima koje su uobičajene u cijelom korpusu. Mjera je
osmišljena sredinom dvadesetog stoljeća u kontekstu informacijskog
pretraživanja, ali je postala standardni alat u širokom rasponu primjena
računalne analize teksta.

Mjera TF-IDF sastoji se od dva komponenta. Prvi komponenta, frekvencija
termina (TF), jednostavno mjeri koliko se često određena riječ
pojavljuje u dokumentu. Može se izraziti kao sirovi broj pojavljivanja
ili, češće, kao relativna frekvencija dobivena dijeljenjem broja
pojavljivanja s ukupnim brojem riječi u dokumentu. Formalno, za riječ
\(t\) u dokumentu \(d\):

\[TF(t, d) = \frac{f_{t,d}}{\sum_{t' \in d} f_{t',d}}\]

U ovoj formuli \(f_{t,d}\) označava broj pojavljivanja riječi \(t\) u
dokumentu \(d\), a nazivnik predstavlja ukupan broj riječi u dokumentu.

Drugi komponenta, inverzna frekvencija dokumenta (IDF), mjeri koliko je
riječ rijetka ili česta u cijelom korpusu. Definira se kao logaritam
omjera ukupnog broja dokumenata i broja dokumenata koji sadrže tu riječ:

\[IDF(t, D) = \ln\left(\frac{N}{|\{d \in D : t \in d\}|}\right)\]

U ovoj formuli \(N\) označava ukupan broj dokumenata u korpusu \(D\), a
nazivnik broji koliko dokumenata sadrži riječ \(t\). Logaritamska
transformacija služi ublažavanju ekstremnih vrijednosti.

Konačna vrijednost TF-IDF dobiva se množenjem ova dva komponenta:

\[TF\text{-}IDF(t, d, D) = TF(t, d) \times IDF(t, D)\]

Promotrimo što ova formula implicira. Riječ koja se često pojavljuje u
jednom dokumentu, ali rijetko u ostalima, imat će visoku vrijednost
TF-IDF jer će oba faktora biti visoka. Riječ koja se pojavljuje u svim
dokumentima imat će IDF jednak nuli jer je logaritam od 1 jednak nuli,
pa će njezina TF-IDF vrijednost također biti nula bez obzira na to
koliko se često pojavljuje. Na taj način mjera automatski neutralizira
utjecaj općenitih, visokofrekventnih riječi.

Tablica 8 prikazuje hipotetski primjer TF-IDF vrijednosti za odabrane
riječi u tri dokumenta o ekonomskoj tematici.

\begin{longtable}[]{@{}
  >{\raggedright\arraybackslash}p{(\linewidth - 6\tabcolsep) * \real{0.0933}}
  >{\centering\arraybackslash}p{(\linewidth - 6\tabcolsep) * \real{0.3200}}
  >{\centering\arraybackslash}p{(\linewidth - 6\tabcolsep) * \real{0.3333}}
  >{\centering\arraybackslash}p{(\linewidth - 6\tabcolsep) * \real{0.2533}}@{}}
\caption{Hipotetske TF-IDF vrijednosti za odabrane riječi u tri
ekonomska teksta}\tabularnewline
\toprule\noalign{}
\begin{minipage}[b]{\linewidth}\raggedright
Riječ
\end{minipage} & \begin{minipage}[b]{\linewidth}\centering
D1 (Fiskalna politika)
\end{minipage} & \begin{minipage}[b]{\linewidth}\centering
D2 (Monetarna politika)
\end{minipage} & \begin{minipage}[b]{\linewidth}\centering
D3 (Opći pregled)
\end{minipage} \\
\midrule\noalign{}
\endfirsthead
\toprule\noalign{}
\begin{minipage}[b]{\linewidth}\raggedright
Riječ
\end{minipage} & \begin{minipage}[b]{\linewidth}\centering
D1 (Fiskalna politika)
\end{minipage} & \begin{minipage}[b]{\linewidth}\centering
D2 (Monetarna politika)
\end{minipage} & \begin{minipage}[b]{\linewidth}\centering
D3 (Opći pregled)
\end{minipage} \\
\midrule\noalign{}
\endhead
\bottomrule\noalign{}
\endlastfoot
proračun & 0.089 & 0.012 & 0.031 \\
deficit & 0.076 & 0.008 & 0.022 \\
kamatna & 0.011 & 0.094 & 0.028 \\
inflacija & 0.023 & 0.081 & 0.035 \\
gospodarstvo & 0.018 & 0.021 & 0.019 \\
\end{longtable}

Iz tablice vidimo da riječi ``proračun'' i ``deficit'' imaju najviše
vrijednosti u dokumentu D1 koji se bavi fiskalnom politikom, dok
``kamatna'' i ``inflacija'' dominiraju u D2 o monetarnoj politici. Riječ
``gospodarstvo'' ima slične, relativno niske vrijednosti u sva tri
dokumenta jer se pojavljuje u svima i nema diskriminacijsku snagu.

\subsection{Matrica supojavljivanja}\label{matrica-supojavljivanja}

Prethodno razmatrane metode reprezentacije fokusirale su se na odnos
između riječi i dokumenata, tretirajući svaku riječ kao nezavisnu
jedinicu. Međutim, značenje riječi u prirodnom jeziku uvelike ovisi o
kontekstu u kojem se pojavljuje i o drugim riječima s kojima se redovito
javlja zajedno. Lingvistička hipoteza distribucijske semantike, koju je
formulirao Zellig Harris sredinom dvadesetog stoljeća, tvrdi da riječi
koje se pojavljuju u sličnim kontekstima imaju slična značenja. Ova
intuicija motivira pristup reprezentaciji teksta temeljen na analizi
supojavljivanja riječi.

Matrica supojavljivanja bilježi koliko se često parovi riječi pojavljuju
zajedno unutar definiranog kontekstualnog prozora. Za razliku od matrice
dokument-termin gdje redci predstavljaju dokumente, u matrici
supojavljivanja i redci i stupci predstavljaju riječi iz vokabulara.
Vrijednost u ćeliji \((i, j)\) označava koliko se puta riječ \(i\)
pojavila u neposrednoj blizini riječi \(j\) u cijelom korpusu. Veličina
kontekstualnog prozora, najčešće definirana kao određeni broj riječi
prije i poslije ciljne riječi, parametar je koji istraživač mora
odrediti ovisno o analitičkim ciljevima.

Zamislimo da analiziramo korpus novinskih članaka o hrvatskom turizmu i
definiramo kontekstualni prozor od dvije riječi s obje strane. Ako se u
tekstu često pojavljuju fraze poput ``turistička sezona'', ``ljetna
sezona'', ``zimska sezona'', matrica supojavljivanja će zabilježiti
visoke vrijednosti za parove (turistička, sezona), (ljetna, sezona) i
(zimska, sezona).

Tablica 9 prikazuje hipotetski isječak matrice supojavljivanja za mali
skup riječi iz korpusa o ekonomiji.

\begin{longtable}[]{@{}lccccc@{}}
\caption{Hipotetski isječak matrice supojavljivanja za ekonomski
vokabular}\tabularnewline
\toprule\noalign{}
& rast & pad & gospodarstvo & inflacija & plaće \\
\midrule\noalign{}
\endfirsthead
\toprule\noalign{}
& rast & pad & gospodarstvo & inflacija & plaće \\
\midrule\noalign{}
\endhead
\bottomrule\noalign{}
\endlastfoot
rast & - & 12 & 87 & 23 & 45 \\
pad & 12 & - & 65 & 34 & 38 \\
gospodarstvo & 87 & 65 & - & 41 & 52 \\
inflacija & 23 & 34 & 41 & - & 28 \\
plaće & 45 & 38 & 52 & 28 & - \\
\end{longtable}

Iz tablice možemo iščitati da se ``rast'' i ``gospodarstvo'' vrlo često
pojavljuju zajedno (87 supojavljivanja), dok ``rast'' i ``pad''
supostoje znatno rjeđe (12), što je očekivano jer se radi o antonimima
koji se rijetko koriste u istom kontekstu. Zanimljivo je primijetiti da
``gospodarstvo'' ima visoke vrijednosti supojavljivanja sa svim ostalim
riječima jer je to općeniti termin koji se prirodno kombinira s raznim
ekonomskim konceptima.

Ključna prednost matrice supojavljivanja jest što omogućuje otkrivanje
semantičkih odnosa između riječi. Riječi koje se pojavljuju u sličnim
kontekstima imat će slične profile supojavljivanja, odnosno slične retke
u matrici. To znači da možemo mjeriti semantičku sličnost između riječi
usporedbom njihovih vektora supojavljivanja.

Tablica 10 pruža usporedni pregled triju metoda reprezentacije teksta.

\begin{longtable}[]{@{}
  >{\raggedright\arraybackslash}p{(\linewidth - 6\tabcolsep) * \real{0.1667}}
  >{\raggedright\arraybackslash}p{(\linewidth - 6\tabcolsep) * \real{0.2292}}
  >{\raggedright\arraybackslash}p{(\linewidth - 6\tabcolsep) * \real{0.2292}}
  >{\raggedright\arraybackslash}p{(\linewidth - 6\tabcolsep) * \real{0.3750}}@{}}
\caption{Usporedba metoda reprezentacije teksta}\tabularnewline
\toprule\noalign{}
\begin{minipage}[b]{\linewidth}\raggedright
Metoda
\end{minipage} & \begin{minipage}[b]{\linewidth}\raggedright
Struktura
\end{minipage} & \begin{minipage}[b]{\linewidth}\raggedright
Što mjeri
\end{minipage} & \begin{minipage}[b]{\linewidth}\raggedright
Tipična primjena
\end{minipage} \\
\midrule\noalign{}
\endfirsthead
\toprule\noalign{}
\begin{minipage}[b]{\linewidth}\raggedright
Metoda
\end{minipage} & \begin{minipage}[b]{\linewidth}\raggedright
Struktura
\end{minipage} & \begin{minipage}[b]{\linewidth}\raggedright
Što mjeri
\end{minipage} & \begin{minipage}[b]{\linewidth}\raggedright
Tipična primjena
\end{minipage} \\
\midrule\noalign{}
\endhead
\bottomrule\noalign{}
\endlastfoot
Vreća riječi (BoW) & Matrica dokument-termin & Frekvencija riječi u
dokumentima & Klasifikacija dokumenata, pretraživanje \\
TF-IDF & Matrica dokument-termin s težinama & Važnost riječi za dokument
u korpusu & Identificiranje ključnih termina, usporedba dokumenata \\
Matrica supojavljivanja & Matrica riječ-riječ & Kontekstualna bliskost
riječi & Semantička analiza, analiza diskursa \\
\end{longtable}

Valja zaključiti da različite metode reprezentacije teksta nude
komplementarne perspektive na tekstualne podatke. Model vreće riječi
pruža jednostavan i robustan temelj za mnoge analitičke zadatke, posebno
za klasifikaciju dokumenata gdje je cilj razlikovati dokumente prema
općenitim tematskim karakteristikama. TF-IDF nadograđuje taj temelj
uvođenjem koncepta težina koje reflektiraju diskriminacijsku vrijednost
riječi, čineći reprezentaciju osjetljivijom na karakteristične termine
pojedinih dokumenata. Matrica supojavljivanja otvara prozor u semantičke
odnose koji definiraju značenje riječi u kontekstu, omogućujući
sofisticiranije analize diskurzivnih obrazaca.

U praksi, izbor metode reprezentacije ovisi o specifičnostima
istraživačkog pitanja, karakteristikama korpusa i računalnim resursima.
Za jednostavne klasifikacijske zadatke na velikim korpusima, TF-IDF
reprezentacija često predstavlja optimalan balans između informativnosti
i računalne učinkovitosti. Za eksplorativne analize semantičkih odnosa i
diskurzivnih obrazaca, matrica supojavljivanja i tehnike koje na njoj
počivaju mogu pružiti bogatije uvide. Sofisticirane studije često
kombiniraju više pristupa, koristeći svaki za aspekte analize kojima
najbolje odgovara.

Napredak u dubokom učenju donio je nove pristupe reprezentaciji teksta
koji nadilaze jednostavne statističke mjere. Vektorske reprezentacije
riječi (word embeddings) poput Word2Vec ili GloVe uče guste vektorske
reprezentacije riječi iz velikih korpusa, zahvaćajući semantičke odnose
na način koji omogućuje aritmetičke operacije nad značenjima. Još
napredniji pristup predstavljaju kontekstualizirane reprezentacije
temeljene na transformerima (BERT, GPT) koje generiraju reprezentacije
ovisne o kontekstu, tako da ista riječ dobiva različitu reprezentaciju
ovisno o rečenici u kojoj se pojavljuje. Ovi pristupi postižu impresivne
rezultate na širokom rasponu zadataka, ali zahtijevaju značajne
računalne resurse i specijalizirano znanje za primjenu.

\section{Pristupi analizi teksta}\label{pristupi-analizi-teksta}

Nakon što smo razmotrili postupke pripreme podataka i metode
reprezentacije teksta, dolazimo do ključnog pitanja: kako iz
pripremljenih tekstualnih podataka izvući smislene zaključke koji će
odgovoriti na naša istraživačka pitanja? Odgovor ovisi o prirodi
problema koji želimo riješiti i o vrsti znanja koje želimo generirati.
Istraživač masovne komunikacije može biti zainteresiran za automatsku
klasifikaciju velikog broja članaka prema temama, za otkrivanje
skrivenih tematskih struktura u korpusu, za mjerenje emocionalnog tona
medijskog izvještavanja ili za identificiranje ključnih aktera u javnom
diskursu. Svaki od ovih zadataka zahtijeva drugačiji analitički pristup,
a izbor metode ima dalekosežne implikacije za vrstu uvida koje možemo
dobiti.

U ovom poglavlju predstavljamo četiri temeljne vrste pristupa analizi
teksta koje se razlikuju prema logici zaključivanja, potrebnim resursima
i vrstama pitanja na koja mogu odgovoriti. Nadzirano strojno učenje
koristi unaprijed označene primjere za treniranje modela koji će
klasificirati nove tekstove u poznate kategorije. Nenadzirano strojno
učenje otkriva latentne strukture u podacima bez prethodnog definiranja
kategorija. Rječnički pristupi oslanjaju se na unaprijed definirane
popise riječi s pridruženim vrijednostima za mjerenje specifičnih
dimenzija teksta poput sentimenta. Ekstrakcija entiteta identificira i
klasificira imenice koje označavaju konkretne objekte iz stvarnog
svijeta poput osoba, organizacija i lokacija.

\subsection{Nadzirano strojno
učenje}\label{nadzirano-strojno-uux10denje}

Zamislimo istraživača koji analizira tisuće komentara objavljenih na
društvenim mrežama tijekom predizborne kampanje. Cilj je kategorizirati
svaki komentar prema tome podržava li određenog kandidata, kritizira ga
ili je neutralan. Ručno kodiranje tolikog broja komentara zahtijevalo bi
mjesece rada i značajne financijske resurse. Nadzirano strojno učenje
nudi alternativu: istraživač ručno kodira relativno mali uzorak
komentara, a zatim koristi te označene primjere za treniranje algoritma
koji će automatski klasificirati preostale komentare. Na taj način
kombiniraju se prednosti ljudske prosudbe s računalnom učinkovitošću.

Proces nadziranog učenja započinje s ručnim označavanjem skupa
dokumenata prema kategorijama od interesa. Ovaj označeni skup dijeli se
na trenažni skup, koji služi za učenje algoritma, i testni skup, koji
služi za evaluaciju performansi. Algoritam analizira karakteristike
dokumenata u trenažnom skupu, tipično njihovu vektorsku reprezentaciju
temeljenu na frekvencijama riječi, i uči pravila koja povezuju te
karakteristike s kategorijama. Naučena pravila potom se primjenjuju na
testni skup, a usporedba predikcija algoritma s pravim oznakama pruža
objektivnu mjeru uspješnosti.

Evaluacija klasifikatora oslanja se na standardne metrike koje
kvantificiraju različite aspekte performansi. Točnost (accuracy) mjeri
udio ispravno klasificiranih dokumenata u ukupnom broju dokumenata:

\[\text{Accuracy} = \frac{TP + TN}{TP + TN + FP + FN}\]

U ovoj formuli \(TP\) označava broj pravilno prepoznatih pozitivnih
slučajeva (true positives), \(TN\) broj pravilno prepoznatih negativnih
slučajeva (true negatives), \(FP\) broj lažno pozitivnih slučajeva
(false positives), a \(FN\) broj lažno negativnih slučajeva (false
negatives).

Preciznost (precision) mjeri udio pravilno klasificiranih pozitivnih
slučajeva među svim slučajevima koje je model klasificirao kao
pozitivne:

\[\text{Precision} = \frac{TP}{TP + FP}\]

Odziv (recall) mjeri udio pravilno klasificiranih pozitivnih slučajeva
među svim stvarno pozitivnim slučajevima:

\[\text{Recall} = \frac{TP}{TP + FN}\]

F1 mjera je harmonijska sredina preciznosti i odziva, pružajući
uravnoteženu mjeru performansi:

\[F_1 = 2 \cdot \frac{\text{Precision} \cdot \text{Recall}}{\text{Precision} + \text{Recall}}\]

Tablica 11 ilustrira interpretaciju ovih metrika na primjeru
klasifikacije komentara.

\begin{longtable}[]{@{}
  >{\raggedright\arraybackslash}p{(\linewidth - 6\tabcolsep) * \real{0.1364}}
  >{\raggedright\arraybackslash}p{(\linewidth - 6\tabcolsep) * \real{0.1364}}
  >{\raggedright\arraybackslash}p{(\linewidth - 6\tabcolsep) * \real{0.4242}}
  >{\raggedright\arraybackslash}p{(\linewidth - 6\tabcolsep) * \real{0.3030}}@{}}
\caption{Evaluacijske metrike za nadzirano učenje}\tabularnewline
\toprule\noalign{}
\begin{minipage}[b]{\linewidth}\raggedright
Metrika
\end{minipage} & \begin{minipage}[b]{\linewidth}\raggedright
Formula
\end{minipage} & \begin{minipage}[b]{\linewidth}\raggedright
Interpretacija u kontekstu
\end{minipage} & \begin{minipage}[b]{\linewidth}\raggedright
Tipična vrijednost
\end{minipage} \\
\midrule\noalign{}
\endfirsthead
\toprule\noalign{}
\begin{minipage}[b]{\linewidth}\raggedright
Metrika
\end{minipage} & \begin{minipage}[b]{\linewidth}\raggedright
Formula
\end{minipage} & \begin{minipage}[b]{\linewidth}\raggedright
Interpretacija u kontekstu
\end{minipage} & \begin{minipage}[b]{\linewidth}\raggedright
Tipična vrijednost
\end{minipage} \\
\midrule\noalign{}
\endhead
\bottomrule\noalign{}
\endlastfoot
Točnost & (TP+TN)/(TP+TN+FP+FN) & Udio ispravno klasificiranih komentara
& 0.70-0.90 \\
Preciznost & TP/(TP+FP) & Pouzdanost kada model kaže ``pozitivan'' &
0.60-0.85 \\
Odziv & TP/(TP+FN) & Koliko pozitivnih komentara model pronalazi &
0.60-0.85 \\
F1 & Harmonijska sredina P i R & Uravnotežena mjera performansi &
0.65-0.85 \\
\end{longtable}

Među algoritmima nadziranog učenja koji se primjenjuju na tekstualne
podatke, logistička regresija, Naive Bayes i Support Vector Machines
(SVM) predstavljaju klasične pristupe koji se i dalje široko koriste.
Logistička regresija modelira vjerojatnost pripadnosti kategoriji kao
logističku funkciju linearne kombinacije značajki, nudeći
interpretabilne koeficijente koji pokazuju doprinos svake riječi
klasifikaciji. Naive Bayes temelji se na Bayesovom teoremu uz
pretpostavku uvjetne nezavisnosti značajki, što je snažna simplifikacija
koja ipak često proizvodi dobre rezultate. SVM traži hiperravninu koja
maksimalno razdvaja kategorije u višedimenzionalnom prostoru značajki.

Praktična primjena nadziranog učenja u istraživanju masovne komunikacije
može se ilustrirati primjerom klasifikacije vijesti prema temama na
hrvatskim portalima. Istraživač započinje definiranjem kategorijalne
sheme, primjerice: politika, gospodarstvo, sport, kultura, crna kronika.
Zatim nasumično odabire uzorak od nekoliko stotina članaka koje ručno
kodira prema definiranim kategorijama. Ovaj označeni skup predstavlja
``zlatni standard'' na temelju kojega će algoritam učiti. Kritično je
osigurati da su kategorije jasno definirane i međusobno isključive, te
da više kodirača postiže visoku razinu slaganja kako bi se osigurala
pouzdanost oznaka.

Nakon označavanja, tekstualne podatke treba transformirati u numeričku
reprezentaciju. Najčešći pristup koristi TF-IDF vrijednosti riječi kao
značajke, stvarajući visokodimenzionalni vektor za svaki dokument.
Algoritam zatim uči statističke obrasce koji povezuju ove vektorske
reprezentacije s kategorijama. Primjerice, model može naučiti da visoka
TF-IDF vrijednost riječi ``proračun'' i ``ministar'' snažno indicira
kategoriju politike, dok visoke vrijednosti za ``pogodak'' i
``utakmica'' indiciraju sport.

Ključno metodološko pitanje jest kako podijeliti podatke za treniranje i
evaluaciju. Standardni pristup koristi unakrsnu validaciju
(cross-validation) gdje se podaci dijele na k podskupova, a model se
trenira k puta, svaki put koristeći različiti podskup za testiranje. Ovo
pruža robusniju procjenu performansi nego jednostavna podjela na
trenažni i testni skup. Za vremenski strukturirane podatke poput
novinarskih članaka, preporučuje se korištenje temporalne validacije
gdje se model trenira na starijim člancima i testira na novijima, što
bolje simulira stvarnu primjenu.

\subsection{Tematsko modeliranje: Latent Dirichlet
Allocation}\label{tematsko-modeliranje-latent-dirichlet-allocation}

Za razliku od nadziranog učenja koje zahtijeva unaprijed definirane
kategorije, tematsko modeliranje pripada skupini nenadziranih metoda
koje otkrivaju latentne strukture u podacima bez prethodne
specifikacije. Najutjecajniji pristup tematskom modeliranju jest Latent
Dirichlet Allocation (LDA), statistički model razvijen početkom dvadeset
i prvog stoljeća koji pretpostavlja da je svaki dokument mješavina
nekoliko apstraktnih tema, a svaka tema karakterizirana distribucijom
preko riječi.

Intuicija iza LDA-e može se ilustrirati primjerom iz novinarstva.
Zamislimo članak na temu klimatskih promjena i energetske politike.
Takav članak vjerojatno sadrži riječi iz različitih tematskih domena:
riječi vezane uz klimu poput ``temperatura'', ``emisije'', ``ugljik'',
riječi vezane uz energetiku poput ``elektrane'', ``obnovljivi'',
``fosilna'', te riječi vezane uz politiku poput ``zakon'',
``regulativa'', ``vlada''. LDA formalno modelira ovu intuiciju
pretpostavljajući da autor dokumenta implicitno odabire mješavinu tema,
a zatim za svaku riječ u dokumentu odabire jednu od tema i generira
riječ prema distribuciji te teme.

Matematički, LDA koristi Dirichletovu distribuciju kao a priori
distribuciju za mješavine tema u dokumentima i za distribucije riječi u
temama. Dirichletova distribucija je multivarijatna generalizacija beta
distribucije i parametrizirana je vektorom koncentracijskih parametara
\(\alpha\). Za dokument \(d\), distribucija tema \(\theta_d\) generira
se iz Dirichletove distribucije:

\[\theta_d \sim \text{Dirichlet}(\alpha)\]

Slično tome, za svaku temu \(k\), distribucija riječi \(\phi_k\)
generira se iz Dirichletove distribucije:

\[\phi_k \sim \text{Dirichlet}(\beta)\]

Generativni proces za svaku riječ \(w_{dn}\) u dokumentu \(d\) uključuje
prvo odabir teme \(z_{dn}\) prema distribuciji tema dokumenta, a zatim
odabir riječi prema distribuciji riječi odabrane teme:

\[z_{dn} \sim \text{Categorical}(\theta_d)\]
\[w_{dn} \sim \text{Categorical}(\phi_{z_{dn}})\]

Inferiranje latentnih tema iz opaženih dokumenata računalno je zahtjevan
problem jer zahtijeva izračun posteriorne distribucije latentnih
varijabli s obzirom na opažene podatke. U praksi se koriste
aproksimativne metode poput Gibbsovog uzorkovanja ili varijacijske
inferecije.

Tablica 12 prikazuje hipotetski primjer rezultata LDA analize na korpusu
hrvatskih političkih vijesti.

\begin{longtable}[]{@{}
  >{\raggedright\arraybackslash}p{(\linewidth - 4\tabcolsep) * \real{0.1304}}
  >{\raggedright\arraybackslash}p{(\linewidth - 4\tabcolsep) * \real{0.5217}}
  >{\raggedright\arraybackslash}p{(\linewidth - 4\tabcolsep) * \real{0.3478}}@{}}
\caption{Primjer rezultata LDA analize na korpusu političkih
vijesti}\tabularnewline
\toprule\noalign{}
\begin{minipage}[b]{\linewidth}\raggedright
Tema
\end{minipage} & \begin{minipage}[b]{\linewidth}\raggedright
Karakteristične riječi
\end{minipage} & \begin{minipage}[b]{\linewidth}\raggedright
Interpretacija
\end{minipage} \\
\midrule\noalign{}
\endfirsthead
\toprule\noalign{}
\begin{minipage}[b]{\linewidth}\raggedright
Tema
\end{minipage} & \begin{minipage}[b]{\linewidth}\raggedright
Karakteristične riječi
\end{minipage} & \begin{minipage}[b]{\linewidth}\raggedright
Interpretacija
\end{minipage} \\
\midrule\noalign{}
\endhead
\bottomrule\noalign{}
\endlastfoot
Tema 1 & proračun, deficit, porez, prihod, rashod & Fiskalna politika \\
Tema 2 & EU, Bruxelles, komisija, fond, članica & Europska politika \\
Tema 3 & škola, učenik, nastavnik, obrazovanje, reforma & Obrazovna
politika \\
Tema 4 & bolnica, liječnik, pacijent, zdravstvo, lista & Zdravstvena
politika \\
Tema 5 & izbori, stranka, kandidat, glasač, kampanja & Izborna
politika \\
\end{longtable}

Praktična primjena LDA-e zahtijeva donošenje nekoliko ključnih odluka.
Istraživač mora odrediti broj tema \(K\) koji nije poznat unaprijed.
Premali broj tema rezultirat će preširokim, heterogenim temama, dok
prevelik broj može fragmentirati koherentne teme i proizvesti teme koje
je teško interpretirati. Ne postoji objektivno ispravna vrijednost jer
optimalan broj ovisi o karakteristikama korpusa i istraživačkim
ciljevima. U praksi se često eksperimentira s različitim vrijednostima i
kombinira kvantitativne mjere poput koherentnosti tema s kvalitativnom
procjenom interpretabilnosti.

Interpretacija tema zahtijeva pažljivu analizu distribucija riječi. Za
svaku temu, istraživač pregledava najvjerojatnije riječi i pokušava
identificirati koherentni koncept koji ih povezuje. Ovaj proces nije
mehanički jer zahtijeva domensko znanje i kontekstualno razumijevanje.
Primjerice, tema s riječima ``bolnica'', ``liječnik'', ``pacijent'',
``zdravstvo'' intuitivno sugerira zdravstvenu politiku. Međutim, tema s
riječima ``projekt'', ``sredstva'', ``program'', ``provedba'' mogla bi
se odnositi na različite domene ovisno o širem kontekstu korpusa.

Za istraživanje masovne komunikacije, LDA omogućuje praćenje tematske
strukture medijskog prostora kroz vrijeme. Istraživač može analizirati
kako se zastupljenost pojedinih tema mijenja, identificirati događaje
koji su potaknuli porast određenih tema ili usporediti tematski fokus
različitih medija. Primjerice, analiza hrvatskih informativnih portala
tijekom pandemije COVID-19 mogla bi otkriti kako je zdravstvena tema
progresivno dominirala medijskim prostorom, potiskujući druge teme, te
kako se s vremenom javljaju podteme vezane uz ekonomske posljedice,
cijepljenje i različite mjere.

Valja napomenuti i ograničenja LDA modela. Model pretpostavlja da su
dokumenti vreće riječi, zanemarujući redoslijed i strukturu teksta. Ne
postoji mehanizam za modeliranje odnosa između tema, poput hijerarhije
ili vremenske dinamike, premda postoje proširenja modela koja adresiraju
ova ograničenja. Konačno, interpretatilnost tema nije garantirana jer
model može producirati statistički koherentne, ali semantički nejasne
teme.

\subsection{Analiza sentimenta}\label{analiza-sentimenta}

Analiza sentimenta, poznata i kao opinion mining, predstavlja skup
metoda za automatsko određivanje emocionalnog tona ili evaluativne
orijentacije teksta. U najjednostavnijem obliku, cilj je klasificirati
tekst kao pozitivan, negativan ili neutralan. Sofisticiranije varijante
mogu mjeriti intenzitet sentimenta na kontinuiranoj skali, razlikovati
više emocionalnih kategorija poput radosti, tuge, straha i ljutnje, ili
identificirati aspekte objekta na koje se sentiment odnosi.

Za istraživanje masovne komunikacije analiza sentimenta pruža alate za
kvantifikaciju emocionalnog tona medijskog izvještavanja i javnog
diskursa. Istraživač može pratiti kako se sentiment prema određenom
političaru mijenja tijekom kampanje, uspoređivati emocionalni ton
izvještavanja različitih medija o istoj temi ili analizirati kako
publika reagira na određene događaje. Takve analize mogu otkriti obrasce
koji bi promakli kvalitativnom čitanju ograničenog broja tekstova.

Rječnički pristupi analizi sentimenta oslanjaju se na unaprijed
definirane popise riječi s pridruženim sentiment vrijednostima. Ideja je
jednostavna: tekst koji sadrži mnogo pozitivnih riječi vjerojatno
izražava pozitivan sentiment, dok tekst bogat negativnim riječima
vjerojatno izražava negativan sentiment. Različiti leksikoni koriste
različite pristupe kodiranju sentimenta. Neki koriste jednostavnu
binarnu klasifikaciju pozitivno/negativno, dok drugi pridružuju
numeričke vrijednosti na skali intenziteta.

Tablica 13 prikazuje usporedbu najčešće korištenih sentimentnih
leksikona.

\begin{longtable}[]{@{}
  >{\raggedright\arraybackslash}p{(\linewidth - 6\tabcolsep) * \real{0.2128}}
  >{\raggedright\arraybackslash}p{(\linewidth - 6\tabcolsep) * \real{0.2766}}
  >{\raggedright\arraybackslash}p{(\linewidth - 6\tabcolsep) * \real{0.3191}}
  >{\raggedright\arraybackslash}p{(\linewidth - 6\tabcolsep) * \real{0.1915}}@{}}
\caption{Usporedba sentimentnih leksikona}\tabularnewline
\toprule\noalign{}
\begin{minipage}[b]{\linewidth}\raggedright
Leksikon
\end{minipage} & \begin{minipage}[b]{\linewidth}\raggedright
Broj riječi
\end{minipage} & \begin{minipage}[b]{\linewidth}\raggedright
Tip kodiranja
\end{minipage} & \begin{minipage}[b]{\linewidth}\raggedright
Primjer
\end{minipage} \\
\midrule\noalign{}
\endfirsthead
\toprule\noalign{}
\begin{minipage}[b]{\linewidth}\raggedright
Leksikon
\end{minipage} & \begin{minipage}[b]{\linewidth}\raggedright
Broj riječi
\end{minipage} & \begin{minipage}[b]{\linewidth}\raggedright
Tip kodiranja
\end{minipage} & \begin{minipage}[b]{\linewidth}\raggedright
Primjer
\end{minipage} \\
\midrule\noalign{}
\endhead
\bottomrule\noalign{}
\endlastfoot
AFINN & \textasciitilde2.500 & Numerička skala (-5 do +5) & ``izvrstan''
= +4, ``grozan'' = -4 \\
Bing & \textasciitilde6.800 & Binarna klasifikacija & ``sjajan'' =
pozitivno, ``loše'' = negativno \\
NRC & \textasciitilde14.000 & Emocije + polaritet & ``smrt'' =
negativno, strah, tuga \\
\end{longtable}

Izračun sentimenta dokumenta najčešće se provodi agregiranjem sentiment
vrijednosti pojedinačnih riječi. Najjednostavniji pristup jednostavno
zbraja vrijednosti svih riječi koje se nalaze u leksikonu, proizvodeci
ukupni sentiment score. Sofisticiranije metode mogu vagati doprinose
prema frekvenciji riječi ili poziciji u dokumentu.

Rječnički pristupi imaju nekoliko značajnih ograničenja. Budući da
tretiraju svaku riječ neovisno o kontekstu, ne mogu uhvatiti nijanse
poput negacije, gdje fraza ``nije loše'' zapravo izražava blago
pozitivan sentiment premda sadrži negativnu riječ ``loše''. Također ne
prepoznaju sarkazam i ironiju, gdje pozitivne riječi mogu biti korištene
za izražavanje negativnog sentimenta. Složene rečenice koje sadrže i
pozitivne i negativne elemente bit će pogrešno svedene na prosjek.

Za hrvatski jezik dodatno ograničenje predstavlja oskudica validiranih
sentimentnih leksikona. Dok za engleski postoje opsežni resursi
razvijani desetljećima, hrvatski leksikoni tipično su manji i manje
temeljito validirani. Projekti poput CroSentiLex razvijaju resurse za
hrvatski, ali istraživači moraju biti svjesni ograničene pokrivenosti i
potencijalne potrebe za prilagodbom specifičnom korpusu.

Osim rječničkih pristupa, postoje i pristupi temeljeni na strojnom
učenju koji tretiraju analizu sentimenta kao klasifikacijski problem.
Model se trenira na korpusu tekstova označenih prema sentimentu i uči
prepoznavati obrasce koji indiciraju pozitivan ili negativan ton. Ovi
pristupi mogu bolje uhvatiti kontekstualne nijanse, uključujući negaciju
i složene evaluativne izraze, no zahtijevaju značajne resurse za
pripremu trenažnih podataka.

U novije vrijeme, pristupi temeljeni na dubokom učenju i velikim
jezičnim modelima postižu impresivne rezultate u analizi sentimenta.
Modeli poput BERT-a, fino podešeni za sentiment klasifikaciju, mogu
razumjeti složene obrasce i kontekstualne ovisnosti koje su bile izvan
dosega ranijih metoda. Za hrvatski jezik dostupni su višejezični modeli
koji, premda nisu optimizirani specifično za hrvatski, često postižu
pristojne rezultate zahvaljujući transferu znanja iz bogatijih jezika.

Primjena analize sentimenta u istraživanju masovne komunikacije može se
ilustrirati analizom komentara čitatelja na hrvatskim informativnim
portalima. Istraživač može pratiti kako se sentiment komentara mijenja u
funkciji teme članka, vremena objave ili političke orijentacije portala.
Takve analize mogu otkriti polarizaciju javnog mnijenja, identificirati
teme koje izazivaju snažne emocionalne reakcije ili pratiti dinamiku
javne rasprave tijekom kontroverznih događaja.

Kritički osvrt na analizu sentimenta mora uključiti razmatranje
valjanosti mjera. Što zapravo mjerimo kada izračunavamo ``sentiment''
teksta? Agregirani score koji kombinira sve riječi u dokumentu može
prikriti složenost evaluativne strukture teksta. Članak koji
uravnoteženo prikazuje i pozitivne i negativne aspekte može imati
neutralan agregirani sentiment, premda zapravo sadrži snažne evaluacije
u oba smjera. Za neke istraživačke svrhe, praćenje proporcije pozitivnih
i negativnih riječi odvojeno može biti informativnije od jedinstvenog
sentiment scorea.

\subsection{Ekstrakcija entiteta}\label{ekstrakcija-entiteta}

U analizi medijskog sadržaja često nas zanima ne samo što mediji govore,
nego i o kome govore. Koji se politički akteri najčešće spominju? Koje
organizacije dominiraju ekonomskim vijestima? Koje lokacije su u fokusu
međunarodnog izvještavanja? Prepoznavanje imenovanih entiteta (Named
Entity Recognition, skraćeno NER) predstavlja tehniku koja automatski
identificira i klasificira imenice koje označavaju konkretne objekte iz
stvarnog svijeta.

Imenovani entiteti obuhvaćaju riječi ili fraze koje se odnose na
jedinstvene objekte koji imaju vlastita imena. Tipične kategorije
uključuju osobe, organizacije, lokacije, vremenske oznake i novčane
vrijednosti. Proces prepoznavanja entiteta konceptualno se sastoji od
dvije faze. Detekcija entiteta identificira dijelove teksta koji
predstavljaju imenice, razlikujući ih od općih imenica i drugih vrsta
riječi. Klasifikacija entiteta zatim svakom detektiranom entitetu
pridružuje kategoriju iz unaprijed definiranog skupa.

Za označavanje entiteta koriste se standardizirane sheme poput BIO
notacije (Begin-Inside-Outside). U ovoj notaciji svaki token dobiva
oznaku koja kombinira poziciju u entitetu i kategoriju entiteta. Token
koji započinje entitet osobe označava se s ``B-PER'', tokeni koji
nastavljaju taj entitet označavaju se s ``I-PER'', dok tokeni koji nisu
dio nijednog entiteta dobivaju oznaku ``O''.

Tablica 14 prikazuje primjer ekstrakcije entiteta iz hipotetskog
novinskog naslova.

\begin{longtable}[]{@{}lll@{}}
\caption{Primjer ekstrakcije entiteta iz naslova ``Premijer Plenković u
Bruxellesu razgovarao s čelnicima EU o 7 milijardi
eura''}\tabularnewline
\toprule\noalign{}
Tekst & Entitet & Kategorija \\
\midrule\noalign{}
\endfirsthead
\toprule\noalign{}
Tekst & Entitet & Kategorija \\
\midrule\noalign{}
\endhead
\bottomrule\noalign{}
\endlastfoot
Premijer Plenković & Plenković & OSOBA \\
u Bruxellesu & Bruxelles & LOKACIJA \\
s čelnicima EU & EU & ORGANIZACIJA \\
o 7 milijardi eura & 7 milijardi eura & NOVČANA VRIJEDNOST \\
u petak & petak & VRIJEME \\
Europska komisija & Europska komisija & ORGANIZACIJA \\
\end{longtable}

Za istraživanje masovne komunikacije ekstrakcija entiteta otvara brojne
analitičke mogućnosti. Analiza vidljivosti aktera može kvantificirati
koliko često se pojedini političari, stranke ili institucije spominju u
medijima i kako se ta vidljivost mijenja kroz vrijeme. Mrežna analiza
može rekonstruirati odnose između entiteta na temelju njihova
supojavljivanja u tekstu.

Ekstrakcija entiteta suočava se s nekoliko izazova koje istraživač mora
uzeti u obzir. Višeznačnost je čest problem jer ista riječ može
označavati entitete različitih kategorija ovisno o kontekstu.
Varijabilnost imenovanja odnosi se na činjenicu da se isti entitet može
pojavljivati pod različitim imenima. Za hrvatski jezik prepoznavanje
entiteta dodatno je otežano morfološkim bogatstvom jer se ime
``Plenković'' pojavljuje u različitim padežima kao ``Plenkovića'',
``Plenkoviću'' i slično.

Tehnike prepoznavanja entiteta razvijale su se kroz nekoliko generacija.
Pristupi temeljeni na pravilima koriste ručno definirane obrasce i
gramatička pravila za identifikaciju entiteta. Primjerice, pravilo može
specificirati da riječ koja počinje velikim slovom nakon titule poput
``gospodin'' ili ``ministrica'' vjerojatno predstavlja ime osobe. Ovakvi
pristupi postižu visoku preciznost za jasno definirane obrasce, ali
imaju ograničen odziv jer ne mogu pokriti sve načine na koje se entiteti
pojavljuju u tekstu.

Statistički pristupi koriste algoritme strojnog učenja trenirane na
označenim podacima. Modeli poput uvjetnih nasumičnih polja (Conditional
Random Fields, CRF) pokazali su se posebno učinkovitima jer mogu
modelirati zavisnosti između susjednih oznaka. Suvremeni sustavi za
prepoznavanje entiteta uglavnom se temelje na dubokom učenju i
neuronskim mrežama, posebno arhitekturama temeljenim na transformerima i
prethodno treniranim jezičnim modelima poput BERT-a koji postižu iznimne
rezultate zahvaljujući sposobnosti hvatanja složenih kontekstualnih
ovisnosti.

Za istraživanje masovne komunikacije, ekstrakcija entiteta omogućuje
konstrukciju mreža aktera na temelju supojavljivanja u medijskim
tekstovima. Ako se dva političara često spominju u istim člancima, to
sugerira da ih mediji percipiraju kao povezane, bilo kroz suradnju ili
sukob. Vizualizacija mreže supojavljivanja može otkriti klastere aktera,
centralne figure i mostove između grupa. Longitudinalna analiza može
pratiti kako se pozicije aktera u mreži mijenjaju kroz vrijeme,
odražavajući promjene u političkom krajoliku.

Tablica 15 pruža usporedni pregled pristupa analizi teksta.

\begin{longtable}[]{@{}
  >{\raggedright\arraybackslash}p{(\linewidth - 6\tabcolsep) * \real{0.1765}}
  >{\raggedright\arraybackslash}p{(\linewidth - 6\tabcolsep) * \real{0.2157}}
  >{\raggedright\arraybackslash}p{(\linewidth - 6\tabcolsep) * \real{0.2549}}
  >{\raggedright\arraybackslash}p{(\linewidth - 6\tabcolsep) * \real{0.3529}}@{}}
\caption{Usporedni pregled pristupa analizi teksta}\tabularnewline
\toprule\noalign{}
\begin{minipage}[b]{\linewidth}\raggedright
Pristup
\end{minipage} & \begin{minipage}[b]{\linewidth}\raggedright
Prednosti
\end{minipage} & \begin{minipage}[b]{\linewidth}\raggedright
Ograničenja
\end{minipage} & \begin{minipage}[b]{\linewidth}\raggedright
Tipična primjena
\end{minipage} \\
\midrule\noalign{}
\endfirsthead
\toprule\noalign{}
\begin{minipage}[b]{\linewidth}\raggedright
Pristup
\end{minipage} & \begin{minipage}[b]{\linewidth}\raggedright
Prednosti
\end{minipage} & \begin{minipage}[b]{\linewidth}\raggedright
Ograničenja
\end{minipage} & \begin{minipage}[b]{\linewidth}\raggedright
Tipična primjena
\end{minipage} \\
\midrule\noalign{}
\endhead
\bottomrule\noalign{}
\endlastfoot
Nadzirano učenje & Visoka preciznost za definirane kategorije &
Zahtijeva označene podatke & Klasifikacija vijesti, detekcija lažnih
vijesti \\
Nenadzirano učenje (LDA) & Otkriva nepoznate strukture & Zahtijeva
interpretaciju & Eksploracija korpusa, praćenje tema \\
Rječnički pristupi & Transparentnost, bez potrebe za treniranjem &
Kontekstualna neosjetljivost & Analiza sentimenta, longitudinalne
studije \\
Ekstrakcija entiteta & Identificira konkretne aktere & Višeznačnost,
varijabilnost & Analiza vidljivosti, mrežna analiza \\
\end{longtable}

Zaključno, svaki od predstavljenih pristupa nudi jedinstvenu perspektivu
na tekstualne podatke i odgovara na različite vrste istraživačkih
pitanja. Nadzirano učenje omogućuje preciznu klasifikaciju prema
unaprijed definiranim kategorijama kada imamo jasnu konceptualnu shemu i
resurse za označavanje primjera. Nenadzirano učenje otkriva latentne
tematske strukture i posebno je korisno u eksplorativnoj fazi
istraživanja kada još ne znamo koje kategorije su relevantne. Rječnički
pristupi kvantificiraju sentiment i emocije na transparentan i
reproducibilan način, idealni za longitudinalne usporedbe. Ekstrakcija
entiteta identificira ključne aktere i omogućuje analize vidljivosti i
relacija. Vješt istraživač kombinira ove pristupe kako bi iz tekstualnih
podataka izvukao bogat spektar uvida o medijskom diskursu i
komunikacijskim praksama, birajući metode koje najbolje odgovaraju
specifičnim istraživačkim pitanjima i dostupnim resursima.

\section{Analiza na razini diskursa}\label{analiza-na-razini-diskursa}

Dosadašnja poglavlja fokusirala su se pretežno na analizu pojedinačnih
riječi ili dokumenata kao cjelina. Međutim, značenje u tekstu rijetko
proizlazi iz izoliranih riječi; ono nastaje kroz odnose između riječi,
kroz načine na koje se riječi kombiniraju u veće jedinice, kroz obrasce
koji se ponavljaju kroz korpus i kroz implicitne strukture koje
organiziraju diskurs. Kada novinar izvještava o ``ekonomskoj krizi'',
značenje te fraze nije jednostavno zbroj značenja riječi ``ekonomska'' i
``kriza'', već emergentna cjelina s vlastitim konotacijama,
asocijacijama i pragmatičkim implikacijama. Slično tome, način na koji
mediji uokviruju određenu temu, koje metafore koriste, koje aktere
stavljaju u prvi plan, a koje marginaliziraju, ima dalekosežne
posljedice za javno razumijevanje društvenih fenomena.

U ovom poglavlju prelazimo s razine pojedinačnih riječi na razinu
diskursa, odnosno na analizu širih jezičnih obrazaca i struktura koje
organiziraju značenje u tekstu. Počinjemo s n-gramima i kolokacijama,
tehnikama koje nam omogućuju da identificiramo uobičajene kombinacije
riječi i izmjerimo snagu njihove asocijacije. Središnji dio poglavlja
posvećujemo analizi okvira, jednoj od najvažnijih metoda u istraživanju
masovne komunikacije, koja ispituje kako mediji konstruiraju određene
interpretacije stvarnosti. Razmatramo i analizu retoričke strukture koja
ispituje kako se dijelovi teksta povezuju u koherentnu cjelinu. Konačno,
predstavljamo mrežnu analizu riječi koja omogućuje vizualizaciju i
kvantifikaciju odnosa između riječi u korpusu.

\subsection{N-grami i kolokacije}\label{n-grami-i-kolokacije}

Dosad smo tekstove analizirali pretežno kroz prizmu pojedinačnih riječi,
tretirajući dokument kao ``vreću riječi'' bez obzira na redoslijed i
međusobne odnose. Međutim, jezik ne funkcionira kao nasumični niz
izoliranih riječi. Riječi se kombiniraju u uobičajene obrasce, neke
kombinacije su češće nego što bismo očekivali na temelju slučajnosti, a
te kombinacije često nose značenje koje nadilazi značenje sastavnih
dijelova. Britanski lingvist John Rupert Firth sažeo je ovu ideju
čuvenom izrekom: ``Riječ ćeš upoznati po društvu koje drži.'' N-grami i
kolokacije predstavljaju tehnike koje nam omogućuju da istražimo to
``društvo'' koje riječi drže.

N-gram je uzastopni niz od n riječi u tekstu. Kada je n jednak 1,
govorimo o unigramima, što odgovara analizi pojedinačnih riječi kakvu
smo već razmatrali. Kada je n jednak 2, govorimo o bigramima ili
parovima uzastopnih riječi. Trigram je niz od tri uzastopne riječi, i
tako dalje. Primjerice, rečenica ``Vlada je usvojila proračun'' sadrži
sljedeće bigrame: ``Vlada je'', ``je usvojila'', ``usvojila proračun''.
Trigrami iste rečenice bili bi: ``Vlada je usvojila'', ``je usvojila
proračun''. Primjetimo da se n-grami preklapaju: riječ ``usvojila''
pojavljuje se i kao drugi element prvog trigrama i kao prvi element
drugog trigrama.

Analiza bigrama otkriva uzastopne parove riječi koji se često pojavljuju
zajedno. U korpusu hrvatskih političkih govora, najčešći bigrami
vjerojatno uključuju kombinacije poput ``Republika Hrvatska'',
``Europska unija'', ``Hrvatski sabor'', ``socijalna pravda'',
``gospodarski rast'' ili ``fiskalna politika''. Mnogi od ovih bigrama
predstavljaju ustaljene fraze ili termine čije je značenje više od
zbroja sastavnih dijelova. ``Europska unija'' nije samo bilo koja unija
koja je europska, već specifična politička i ekonomska zajednica s jasno
definiranim značenjem.

Tablica 16 prikazuje primjere najčešćih bigrama iz hipotetskog korpusa
hrvatskih političkih govora.

\begin{longtable}[]{@{}lcl@{}}
\caption{Primjeri čestih bigrama u korpusu političkih
govora}\tabularnewline
\toprule\noalign{}
Bigram & Frekvencija & Tip fraze \\
\midrule\noalign{}
\endfirsthead
\toprule\noalign{}
Bigram & Frekvencija & Tip fraze \\
\midrule\noalign{}
\endhead
\bottomrule\noalign{}
\endlastfoot
Republika Hrvatska & 2.847 & Službeni naziv \\
Europska unija & 1.923 & Politička organizacija \\
Hrvatski sabor & 1.456 & Institucija \\
gospodarski rast & 987 & Ekonomski termin \\
javni interes & 756 & Pravni pojam \\
socijalna pravda & 612 & Ideološki koncept \\
\end{longtable}

Međutim, kao i kod analize pojedinačnih riječi, najčešći bigrami često
nisu najinformativniji. Dominiraju kombinacije funkcionalnih riječi
poput ``u kojima'', ``za koje'', ``da se'', ``koji su'' i slično. Ove
kombinacije su česte, ali rijetko nose značajan sadržaj za većinu
istraživačkih pitanja. Zato se primjenjuju slične tehnike filtriranja
kao kod pojedinačnih riječi: uklanjanje bigrama koji sadrže zaustavne
riječi, fokusiranje na bigrame koji sadrže imenice, pridjeve ili
glagole, ili korištenje statističkih mjera koje identificiraju bigrame s
neočekivano visokom frekvencijom.

Kolokacija je širi pojam od bigrama. Dok bigram zahtijeva da riječi budu
neposredno susjedne, kolokacija označava tendenciju dviju riječi da se
pojavljuju u međusobnoj blizini, unutar određenog prozora konteksta, čak
i ako nisu neposredno susjedne. Primjerice, riječi ``donijeti'' i
``odluka'' čine kolokaciju u hrvatskom jeziku: one se često pojavljuju
zajedno u frazi ``donijeti odluku'', ali mogu biti razdvojene drugim
riječima u konstrukcijama poput ``donijeti važnu odluku'' ili ``donijeti
konačnu političku odluku''. Kolokacije odražavaju idiomatske i
konvencionalne načine izražavanja u jeziku.

Za identifikaciju statistički značajnih kolokacija koriste se razne
mjere asocijacije koje kvantificiraju snagu veze između dviju riječi.
Ove mjere uspoređuju opaženu frekvenciju zajedničkog pojavljivanja s
očekivanom frekvencijom pod pretpostavkom nezavisnosti. Ako se dvije
riječi pojavljuju zajedno češće nego što bismo očekivali na temelju
njihovih individualnih frekvencija, to sugerira da postoji neka veza
između njih.

Uzajamna informacija (Pointwise Mutual Information, PMI) jedna je od
najčešće korištenih mjera. Definira se kao:

\[PMI(w_1, w_2) = \log_2 \frac{P(w_1, w_2)}{P(w_1) \cdot P(w_2)}\]

U ovoj formuli \(P(w_1, w_2)\) označava vjerojatnost zajedničkog
pojavljivanja riječi \(w_1\) i \(w_2\) unutar definiranog prozora, a
\(P(w_1)\) i \(P(w_2)\) označavaju marginalne vjerojatnosti pojedinih
riječi. Vjerojatnosti se procjenjuju iz korpusa kao relativne
frekvencije. Ako je PMI jednak nuli, riječi su nezavisne, odnosno
pojavljuju se zajedno točno onoliko često koliko bismo očekivali na
temelju slučajnosti. Pozitivne vrijednosti PMI ukazuju na pozitivnu
asocijaciju gdje se riječi privlače, dok negativne vrijednosti ukazuju
na negativnu asocijaciju gdje se riječi međusobno izbjegavaju. Tipične
jake kolokacije imaju PMI vrijednosti između 3 i 10.

T-vrijednost (t-score) predstavlja alternativnu mjeru koja uzima u obzir
varijabilnost podataka:

\[t = \frac{O - E}{\sqrt{O}}\]

U ovoj formuli \(O\) označava opaženu frekvenciju kolokacije, a \(E\)
očekivanu frekvenciju pod pretpostavkom nezavisnosti. T-vrijednost je
konzervativnija od PMI i favorizira česte kolokacije. Visoke
t-vrijednosti tipično imaju kombinacije čestih riječi koje se redovito
pojavljuju zajedno, dok rijetke ali snažno asocirane kombinacije mogu
imati niže t-vrijednosti.

Tablica 17 uspoređuje karakteristike različitih mjera asocijacije.

\begin{longtable}[]{@{}
  >{\raggedright\arraybackslash}p{(\linewidth - 6\tabcolsep) * \real{0.1667}}
  >{\raggedright\arraybackslash}p{(\linewidth - 6\tabcolsep) * \real{0.2143}}
  >{\raggedright\arraybackslash}p{(\linewidth - 6\tabcolsep) * \real{0.3810}}
  >{\raggedright\arraybackslash}p{(\linewidth - 6\tabcolsep) * \real{0.2381}}@{}}
\caption{Usporedba mjera asocijacije za identifikaciju
kolokacija}\tabularnewline
\toprule\noalign{}
\begin{minipage}[b]{\linewidth}\raggedright
Mjera
\end{minipage} & \begin{minipage}[b]{\linewidth}\raggedright
Formula
\end{minipage} & \begin{minipage}[b]{\linewidth}\raggedright
Karakteristike
\end{minipage} & \begin{minipage}[b]{\linewidth}\raggedright
Primjena
\end{minipage} \\
\midrule\noalign{}
\endfirsthead
\toprule\noalign{}
\begin{minipage}[b]{\linewidth}\raggedright
Mjera
\end{minipage} & \begin{minipage}[b]{\linewidth}\raggedright
Formula
\end{minipage} & \begin{minipage}[b]{\linewidth}\raggedright
Karakteristike
\end{minipage} & \begin{minipage}[b]{\linewidth}\raggedright
Primjena
\end{minipage} \\
\midrule\noalign{}
\endhead
\bottomrule\noalign{}
\endlastfoot
PMI & \(\log_2 \frac{P(w_1, w_2)}{P(w_1) \cdot P(w_2)}\) & Osjetljiva na
rijetke riječi & Specifični termini, idiomi \\
T-vrijednost & \(\frac{O - E}{\sqrt{O}}\) & Favorizira česte kombinacije
& Uobičajeni jezični obrasci \\
Log-likelihood & \(2 \sum O \cdot \ln\frac{O}{E}\) & Robusna,
statistički testabilna & Opća primjena \\
\end{longtable}

U kontekstu istraživanja masovne komunikacije, analiza n-grama i
kolokacija može odgovoriti na niz pitanja. Koje su karakteristične fraze
pojedinih političkih stranaka ili političara? Istraživač može usporediti
bigrame i trigrame u govorima različitih aktera i identificirati jezične
markere koji razlikuju njihov diskurs. Kako se kolokacije određene
riječi razlikuju između medija? Primjerice, koje riječi mediji
različitih političkih orijentacija povezuju s riječju ``migranti''?
Jedan medij možda preferira kolokacije poput ``ilegalni migranti'' i
``migrantska kriza'', dok drugi preferira ``zaštita izbjeglica'' i
``humanitarna pomoć''. Ove razlike u kolokacijama odražavaju različite
interpretativne okvire koje mediji primjenjuju na istu temu.

Longitudinalna analiza kolokacija može pratiti promjene u jezičnoj
upotrebi kroz vrijeme. Koje su se nove kolokacije pojavile s riječju
``klima'' u posljednjem desetljeću? Vjerojatno bismo vidjeli porast
kolokacija poput ``klimatska kriza'', ``klimatske promjene'',
``klimatska neutralnost'' koje odražavaju promjene u javnom diskursu o
okolišu. Slično tome, analiza kolokacija riječi ``digitalni'' kroz
vrijeme mogla bi otkriti pomak od tehničkih kolokacija poput ``digitalni
signal'' prema društvenim kolokacijama poput ``digitalna pismenost'',
``digitalna transformacija'' i ``digitalni jaz''.

Kontekstualna analiza sentimenta predstavlja posebno važnu primjenu
n-grama. Jednostavni rječnički pristupi sentimentu tretiraju svaku riječ
neovisno o kontekstu, što dovodi do pogrešaka kod negacija. Rečenica
``Film nije bio loš'' izražava blago pozitivan sentiment, no jednostavni
pristup prepoznaje samo riječ ``loš'' i klasificira rečenicu kao
negativnu. Analiza bigrama omogućuje da identificiramo obrasce poput
``nije loš'', ``nije dobro'', ``nikad sretniji'' i tretiramo ih kao
cjeline s vlastitim sentimentom. Slično tome, pojačivači poput ``vrlo''
i ``izrazito'' te ublažavači poput ``pomalo'' i ``donekle'' modificiraju
sentiment susjednih riječi na načine koje možemo uhvatiti analizom
bigrama.

\subsection{Analiza okvira}\label{analiza-okvira}

Uokvirivanje (framing) predstavlja jedan od središnjih koncepata u
istraživanju masovne komunikacije. Ideja je da mediji ne samo prenose
informacije o događajima, nego ih aktivno interpretiraju, odabirući
određene aspekte stvarnosti, naglašavajući ih i čineći ih istaknutijima,
dok druge aspekte marginaliziraju ili potpuno ignoriraju. Robert Entman,
jedan od najutjecajnijih teoretičara uokvirivanja, definirao je ovaj
koncept na način koji je postao kanonski u literaturi: uokvirivanje je
proces ``odabira nekih aspekata percipirane stvarnosti i činjenja tih
aspekata istaknutijima u komunikacijskom tekstu, na način da se
promovira određena definicija problema, kauzalna interpretacija, moralna
evaluacija i/ili preporuka tretmana.''

Entmanova definicija identificira četiri ključne funkcije okvira.
Definicija problema određuje o čemu se zapravo radi, što je u igri, tko
ili što je pogođeno. Kauzalna interpretacija identificira uzroke
problema, pripisujući odgovornost određenim akterima, silama ili
okolnostima. Moralna evaluacija procjenjuje aktere i njihove postupke u
terminima dobrog i lošeg, pravednog i nepravednog. Preporuka tretmana
sugerira rješenja, akcije koje bi trebalo poduzeti, politike koje bi
trebalo implementirati.

Klasičan primjer koji ilustrira moć uokvirivanja jest izvještavanje o
demonstracijama. Isti događaj može biti uokviren kao ``mirni prosvjed
građana za svoja prava'' ili kao ``nasilni neredi koji ugrožavaju javni
red''. Prvi okvir definira problem kao nepoštivanje građanskih prava,
uzrok pripisuje vlastima koje ta prava krše, moralno evaluira
prosvjednike kao legitimne aktere i implicira da bi vlasti trebale
odgovoriti na zahtjeve. Drugi okvir definira problem kao narušavanje
javnog reda, uzrok pripisuje prosvjednicima, moralno ih evaluira kao
prijetnju i implicira da bi policija trebala uspostaviti red. Odabir
okvira utječe na to kako će publika razumjeti događaj, koje uzroke će
pripisati, koje aktere će smatrati odgovornima i koje reakcije će
smatrati prikladnima.

Tablica 18 prikazuje Entmanove funkcije okvira s primjerima iz hrvatskog
medijskog konteksta.

\begin{longtable}[]{@{}
  >{\raggedright\arraybackslash}p{(\linewidth - 4\tabcolsep) * \real{0.3148}}
  >{\raggedright\arraybackslash}p{(\linewidth - 4\tabcolsep) * \real{0.2222}}
  >{\raggedright\arraybackslash}p{(\linewidth - 4\tabcolsep) * \real{0.4630}}@{}}
\caption{Entmanove funkcije okvira primijenjene na temu
migracija}\tabularnewline
\toprule\noalign{}
\begin{minipage}[b]{\linewidth}\raggedright
Funkcija okvira
\end{minipage} & \begin{minipage}[b]{\linewidth}\raggedright
Definicija
\end{minipage} & \begin{minipage}[b]{\linewidth}\raggedright
Primjer: tema migracija
\end{minipage} \\
\midrule\noalign{}
\endfirsthead
\toprule\noalign{}
\begin{minipage}[b]{\linewidth}\raggedright
Funkcija okvira
\end{minipage} & \begin{minipage}[b]{\linewidth}\raggedright
Definicija
\end{minipage} & \begin{minipage}[b]{\linewidth}\raggedright
Primjer: tema migracija
\end{minipage} \\
\midrule\noalign{}
\endhead
\bottomrule\noalign{}
\endlastfoot
Definicija problema & Što je u pitanju? & ``Sigurnosna prijetnja''
vs.~``Humanitarna kriza'' \\
Kauzalna interpretacija & Tko/što je uzrok? & ``Krijumčari''
vs.~``Ratovi i siromaštvo'' \\
Moralna evaluacija & Tko je dobar/loš? & ``Ilegalni uljezi''
vs.~``Ranjive osobe'' \\
Preporuka tretmana & Što treba učiniti? & ``Zatvoriti granice''
vs.~``Pružiti zaštitu'' \\
\end{longtable}

Računalna analiza okvira pokušava automatizirati ili barem podržati
identifikaciju okvira u velikim korpusima teksta. Ovo je izazovan
zadatak jer okviri nisu eksplicitno označeni u tekstu, već proizlaze iz
kompleksne interakcije leksičkih izbora, retoričkih struktura, metafora,
narativnih obrazaca i implicitnih pretpostavki. Ipak, određene tehnike
mogu pomoći u identifikaciji elemenata koji konstituiraju okvire.

Leksički pristupi polaze od pretpostavke da se okviri manifestiraju kroz
karakteristične riječi i fraze. Istraživač može definirati ``rječnike
okvira'' koji sadrže riječi tipične za pojedine okvire, analogno
sentimentnim rječnicima. Primjerice, okvir ``ekonomske racionalnosti''
mogao bi uključivati riječi poput ``učinkovitost'', ``troškovi'',
``konkurentnost'', ``produktivnost'', ``ulaganje'', dok okvir ``ljudskih
prava'' mogao bi uključivati ``dostojanstvo'', ``zaštita'',
``jednakost'', ``diskriminacija'', ``pravednost''. Prebrajanjem riječi
iz svakog rječnika može se procijeniti relativna prominentnost
različitih okvira u tekstu ili korpusu.

Kolokacijska analiza pruža sofisticiraniji pristup identifikaciji
okvira. Umjesto da prebrojimo pojedinačne riječi, analiziramo koje
riječi se tipično pojavljuju zajedno s ključnim pojmovima. Ako nas
zanima kako je uokvirena tema migracija, možemo analizirati kolokacije
riječi ``migranti'', ``izbjeglice'', ``azilanti'' i srodnih termina.
Usporedba kolokacija između različitih medija ili vremenskih razdoblja
može otkriti sistematske razlike u uokvirivanju. Istraživač može
izračunati log-likelihood omjere za kolokacije u dva korpusa i
identificirati kolokacije koje su statistički značajno češće u jednom
korpusu nego u drugom. Ove ``karakteristične kolokacije'' predstavljaju
empirijsku osnovu za identifikaciju različitih interpretativnih okvira.

Analiza metafora predstavlja posebno važan pristup jer metafore često
čine srž konceptualnih okvira. Kognitivni lingvisti George Lakoff i Mark
Johnson pokazali su da metafore nisu samo stilsko ukrašavanje, već
temeljni način na koji konceptualiziramo apstraktne pojmove. Metafora
``EKONOMIJA JE ZDRAVLJE'' strukturira ekonomski diskurs kroz pojmove
poput ``zdravog gospodarstva'', ``bolesne ekonomije'', ``oporavka'',
``dijagnoze''. Automatska identifikacija metafora u tekstu ostaje
izazovan zadatak, no kolokacije koje kombiniraju apstraktne i konkretne
domene mogu ukazivati na metaforički jezik.

Istraživač koji koristi računalne metode za analizu okvira mora biti
svjestan da one pružaju indikatore, a ne definitivne identifikacije
okvira. Kolokacije, teme i leksičke frekvencije mogu ukazati na obrasce
koji sugeriraju određene okvire, no konačna interpretacija zahtijeva
kvalitativnu analizu konkretnih tekstova. Najbolja praksa kombinira
računalno potpomognutu identifikaciju obrazaca s detaljnom analizom
reprezentativnih primjera koji ilustriraju i validiraju te obrasce.

\subsection{Analiza retoričke
strukture}\label{analiza-retoriux10dke-strukture}

Tekst nije samo skup rečenica poredanih jedna za drugom; rečenice su
međusobno povezane odnosima koji čine tekst koherentnom cjelinom.
Teorija retoričke strukture (Rhetorical Structure Theory, RST), koju su
razvili William Mann i Sandra Thompson, pruža okvir za analizu tih
odnosa. Prema RST-u, tekst se može rastaviti na elementarne diskurzivne
jedinice, obično klauzule, koje su povezane retoričkim relacijama poput
uzroka, posljedice, kontrasta, elaboracije, dokaza i sličnih.

Razmotrimo primjer iz političkog izvještavanja: ``Vlada je povećala
porez na dohodak. To će negativno utjecati na potrošnju građana.'' Dvije
rečenice povezane su relacijom uzrok-posljedica: prva rečenica navodi
uzrok (povećanje poreza), a druga posljedicu (pad potrošnje).
Alternativna struktura mogla bi biti: ``Vlada je povećala porez na
dohodak, iako je obećala suprotno.'' Ovdje je relacija kontrast ili
koncesija: druga klauzula uvodi informaciju koja je u napetosti s
očekivanjem koje proizlazi iz prve.

RST razlikuje nukleus i satelit u većini relacija. Nukleus je centralni,
nezavisni dio koji može stajati sam, dok je satelit periferni dio koji
dopunjuje, objašnjava ili modificira nukleus. U relaciji dokaza, tvrdnja
je nukleus, a dokaz je satelit koji podupire tu tvrdnju. U relaciji
elaboracije, općenita izjava je nukleus, a specifičniji detalji su
sateliti. Neke relacije su simetrične poput kontrasta ili slijeda, gdje
oba dijela imaju jednak status.

Tablica 19 prikazuje tipove diskurzivnih relacija i njihove leksičke
signale.

\begin{longtable}[]{@{}
  >{\raggedright\arraybackslash}p{(\linewidth - 4\tabcolsep) * \real{0.2642}}
  >{\raggedright\arraybackslash}p{(\linewidth - 4\tabcolsep) * \real{0.3208}}
  >{\raggedright\arraybackslash}p{(\linewidth - 4\tabcolsep) * \real{0.4151}}@{}}
\caption{Tipovi diskurzivnih relacija i njihovi leksički
signali}\tabularnewline
\toprule\noalign{}
\begin{minipage}[b]{\linewidth}\raggedright
Tip relacije
\end{minipage} & \begin{minipage}[b]{\linewidth}\raggedright
Tipični veznici
\end{minipage} & \begin{minipage}[b]{\linewidth}\raggedright
Diskurzivna funkcija
\end{minipage} \\
\midrule\noalign{}
\endfirsthead
\toprule\noalign{}
\begin{minipage}[b]{\linewidth}\raggedright
Tip relacije
\end{minipage} & \begin{minipage}[b]{\linewidth}\raggedright
Tipični veznici
\end{minipage} & \begin{minipage}[b]{\linewidth}\raggedright
Diskurzivna funkcija
\end{minipage} \\
\midrule\noalign{}
\endhead
\bottomrule\noalign{}
\endlastfoot
Kauzalna & jer, zato, stoga, budući da & Objašnjenje, argumentacija \\
Kontrastna & ali, međutim, ipak, no & Suprotstavljanje, koncesija \\
Aditivna & i, također, osim toga, nadalje & Nadogradnja, nabrajanje \\
Temporalna & zatim, potom, nakon toga, prije & Sekvenciranje,
naracija \\
Kondicijska & ako, ukoliko, pod uvjetom & Hipotetički scenariji \\
\end{longtable}

Za istraživače masovne komunikacije, čak i djelomična analiza retoričke
strukture može biti vrijedna. Distribucija relacija u korpusu može
otkriti karakteristike diskursa. Tekstovi bogati relacijama dokaza i
uzroka mogu sugerirati argumentativni, analitički stil. Tekstovi s
dominantnim relacijama slijeda i elaboracije mogu sugerirati narativni,
deskriptivni stil. Usporedba distribucije relacija između žanrova,
medija ili autora može otkriti sistematske razlike u diskurzivnim
praksama.

Analiza diskurzivnih veznika predstavlja jednostavniji pristup koji ne
zahtijeva potpunu RST analizu. Veznici poput ``jer'', ``zato'',
``stoga'' signaliziraju kauzalne relacije. Veznici poput ``ali'',
``međutim'', ``ipak'' signaliziraju kontrastne relacije. Prebrajanjem i
analizom distribucije ovih veznika može se dobiti gruba slika retoričke
organizacije teksta. Primjena u analizi medijskih tekstova može
uključivati usporedbu argumentativne strukture izvještavanja različitih
medija o istoj temi. Koriste li određeni mediji više kauzalnih
objašnjenja, dok se drugi oslanjaju na puko navođenje činjenica? Kako se
struktura argumentacije razlikuje u vijestima, komentarima i analizama?

\subsection{Mreže riječi i
vizualizacija}\label{mreux17ee-rijeux10di-i-vizualizacija}

Dosadašnje tehnike koje smo razmatrali produciraju tablice frekvencija,
mjera asocijacije i distribucija. Međutim, odnosi između riječi u tekstu
inherentno su mrežni: riječi su povezane s drugim riječima kroz
kolokacije, supojavljivanja i semantičke odnose. Mrežna analiza riječi
eksplicira ovu strukturu, omogućujući vizualizaciju i kvantifikaciju
kompleksnih obrazaca koji bi inače ostali skriveni.

U mreži riječi, čvorovi predstavljaju riječi, a veze (bridovi)
predstavljaju odnose između riječi. Veze mogu biti definirane na
različite načine: bigram relacije gdje riječ A neposredno prethodi
riječi B, kolokacijske relacije gdje se riječi pojavljuju zajedno unutar
prozora, korelacijske relacije gdje se riječi pojavljuju u istim
dokumentima, ili semantičke relacije gdje su riječi sinonimi ili
hiperonimi. Veze mogu biti usmjerene ili neusmjerene, te mogu imati
težine koje odražavaju snagu asocijacije.

Mrežna analiza omogućuje izračun raznih mrežnih metrika koje
kvantificiraju strukturu diskursa. Stupanj čvora (degree) mjeri broj
veza koje čvor ima, ukazujući na ``povezanost'' riječi. Riječi s visokim
stupnjem su centralne u diskursu, povezujući mnoge druge riječi.
Međupoloženost (betweenness centrality) mjeri koliko često čvor leži na
najkraćim putevima između drugih čvorova, ukazujući na ``mostovnu''
ulogu riječi koja povezuje inače odvojene dijelove mreže.

Formalno, međupoloženost čvora \(v\) definira se kao:

\[B(v) = \sum_{s \neq v \neq t} \frac{\sigma_{st}(v)}{\sigma_{st}}\]

U ovoj formuli \(\sigma_{st}\) označava broj najkraćih puteva između
čvorova \(s\) i \(t\), a \(\sigma_{st}(v)\) broj tih puteva koji prolaze
kroz \(v\). Visoka međupoloženost ukazuje da je riječ ključna za
povezivanje različitih dijelova diskursa.

Detekcija zajednica (community detection) identificira skupine čvorova
koje su gusto povezane međusobno, a slabo povezane s ostatkom mreže.
Algoritmi poput Louvainove metode mogu automatski identificirati ove
zajednice. U mreži riječi, zajednice često odgovaraju tematskim ili
semantičkim skupinama. Primjerice, u mreži konstruiranoj iz političkih
govora, jedna zajednica mogla bi sadržavati riječi vezane uz ekonomiju
poput proračun, deficit, porez i rast, druga riječi vezane uz socijalnu
politiku poput mirovina, zdravstvo i obrazovanje, a treća riječi vezane
uz vanjsku politiku poput EU, NATO i diplomatija.

Tablica 20 prikazuje mrežne metrike i njihovu interpretaciju u analizi
diskursa.

\begin{longtable}[]{@{}
  >{\raggedright\arraybackslash}p{(\linewidth - 4\tabcolsep) * \real{0.1875}}
  >{\raggedright\arraybackslash}p{(\linewidth - 4\tabcolsep) * \real{0.2500}}
  >{\raggedright\arraybackslash}p{(\linewidth - 4\tabcolsep) * \real{0.5625}}@{}}
\caption{Mrežne metrike i njihova interpretacija u analizi
diskursa}\tabularnewline
\toprule\noalign{}
\begin{minipage}[b]{\linewidth}\raggedright
Metrika
\end{minipage} & \begin{minipage}[b]{\linewidth}\raggedright
Definicija
\end{minipage} & \begin{minipage}[b]{\linewidth}\raggedright
Interpretacija u diskursu
\end{minipage} \\
\midrule\noalign{}
\endfirsthead
\toprule\noalign{}
\begin{minipage}[b]{\linewidth}\raggedright
Metrika
\end{minipage} & \begin{minipage}[b]{\linewidth}\raggedright
Definicija
\end{minipage} & \begin{minipage}[b]{\linewidth}\raggedright
Interpretacija u diskursu
\end{minipage} \\
\midrule\noalign{}
\endhead
\bottomrule\noalign{}
\endlastfoot
Stupanj & Broj veza čvora & Povezanost, prominentnost \\
Međupoloženost & Broj najkraćih puteva kroz čvor & Mostovna uloga,
povezivanje tema \\
Bliskost & Prosječna udaljenost do svih čvorova & Globalna
centralnost \\
Koeficijent klasteriranja & Udio povezanih susjeda & Lokalna kohezija \\
\end{longtable}

U kontekstu istraživanja masovne komunikacije, mreže riječi mogu
vizualizirati i kvantificirati strukturu diskursa na načine koji
nadilaze jednostavne liste frekvencija. Usporedba mreža između
različitih medija može otkriti sistematske razlike u načinu na koji
organiziraju i povezuju koncepte. Dva medija mogu koristiti slične
riječi, ali ih povezivati na različite načine, što se manifestira u
različitoj topologiji njihovih mreža. Dinamička analiza mreža može
pratiti kako se struktura diskursa mijenja kroz vrijeme,
identificirajući pojavu novih klastera, nestanak starih veza ili
promjene u centralnosti pojedinih pojmova.

Mreže supojavljivanja entiteta predstavljaju posebno korisnu primjenu za
istraživanje masovne komunikacije. Ako povežemo osobe koje se spominju u
istim člancima, dobivamo mrežu koja odražava medijsku percepciju odnosa
između aktera. Političari koji se često spominju zajedno mogu biti
percipirani kao saveznici ili protivnici; analiza konteksta može
razlikovati ove dvije interpretacije. Centralni akteri u mreži su oni
koji povezuju različite sfere javnog života, dok se periferni akteri
pojavljuju samo u specifičnim kontekstima. Analiza promjena u mreži kroz
vrijeme može otkriti kako se politički krajolici restrukturiraju, kako
novi akteri ulaze u centar pozornosti, a stari marginaliziraju.

Vizualizacija mreže riječi može biti moćan komunikacijski alat za
prezentaciju nalaza istraživanja, no zahtijeva pažljivu izradu.
Prezasićene mreže s previše čvorova i veza postaju nečitljive.
Preporučuje se ograničiti mrežu na nekoliko stotina najvažnijih čvorova,
koristiti veličinu čvorova proporcionalnu njihovoj centralnosti, boju
čvorova za označavanje zajednica i debljinu veza proporcionalnu snazi
asocijacije. Interaktivne vizualizacije koje omogućuju zumiranje i
filtriranje mogu pomoći u eksploraciji kompleksnih mreža.

Integracija različitih tehnika diskurzivne analize omogućuje bogatiji
uvid nego što bi svaka pojedinačno mogla pružiti. Tipičan integrirani
radni tijek započinje eksplorativnom analizom kolokacija i mreža riječi
kako bi se stekao uvid u opću strukturu korpusa. Na temelju tih uvida
razvijaju se specifične hipoteze o okvirima ili diskurzivnim
strategijama. Zatim se koristi ciljana analiza kolokacija određenih
ključnih riječi, distribucije diskurzivnih veznika ili sentimentnih
obrazaca kako bi se kvantificirali relevantni obrasci. Konačno, provodi
se kvalitativna analiza reprezentativnih tekstova koji ilustriraju
identificirane obrasce, validirajući kvantitativne nalaze i pružajući
dublje razumijevanje diskurzivnih mehanizama.

\section{Ograničenja metode}\label{ograniux10denja-metode}

Prethodna poglavlja predstavila su impresivan arsenal tehnika za
računalnu analizu teksta: od pripreme podataka i reprezentacije, preko
klasifikacije i tematskog modeliranja, do sofisticirane diskurzivne
analize. Čitatelj bi mogao steći dojam da su ove metode gotovo magične u
svojoj sposobnosti da iz velikih količina nestrukturiranog teksta izvuku
smislene uvide o društvenim fenomenima. Međutim, svaka metoda ima svoja
ograničenja, a kritičko razumijevanje tih ograničenja nužno je za
odgovorno i valjano istraživanje. Kao što su Grimmer, Roberts i Stewart
upozorili, automatske metode analize teksta ``nisu zamjena za pažljivo
razmišljanje i pomno čitanje te zahtijevaju opsežnu i problemski
specifičnu validaciju.''

\subsection{Valjanost i pouzdanost}\label{valjanost-i-pouzdanost}

Valjanost i pouzdanost tradicionalni su kriteriji kvalitete mjerenja u
društvenim znanostima, a njihova primjena na računalnu analizu teksta
zahtijeva pažljivu konceptualizaciju.

Valjanost mjerenja odnosi se na pitanje mjeri li naša mjera ono što
mislimo da mjeri. U kontekstu analize teksta, valjanost ima nekoliko
dimenzija. Sadržajna valjanost pita pokriva li naša operacionalizacija
sve relevantne aspekte koncepta koji želimo mjeriti. Primjerice, ako
mjerimo sentiment teksta prebrojavanjem pozitivnih i negativnih riječi
iz rječnika, pokriva li taj rječnik sve relevantne načine izražavanja
sentimenta u našem korpusu? Propuštamo li ironiju, sarkazam, implicitne
evaluacije? Konstruktna valjanost pita odgovara li naša mjera teorijskom
konstruktu koji je u pozadini. Ako koristimo tematsko modeliranje za
identifikaciju ``tema'' u korpusu, odgovaraju li statistički
identificirane teme onome što teorija komunikacije podrazumijeva pod
``temama''? Prediktivna valjanost pita predviđa li naša mjera druge
varijable s kojima bi teorijski trebala biti povezana.

Problem valjanosti posebno je akutan kod prijenosa metoda i resursa iz
jednog konteksta u drugi. Sentimentni rječnik razvijen za analizu
recenzija proizvoda na engleskom možda neće valjano mjeriti sentiment u
hrvatskim političkim komentarima. Klasifikator treniran na američkim
vijestima možda neće valjano kategorizirati hrvatske vijesti. Svaki
prijenos zahtijeva revalidaciju u novom kontekstu.

Pouzdanost odnosi se na konzistentnost mjerenja. Test-retest pouzdanost
pita hoće li ponovljena primjena iste metode na iste podatke proizvesti
iste rezultate. Za determinističke algoritme poput prebrajanja riječi iz
rječnika ovo je trivijalno zadovoljeno. Međutim, mnogi suvremeni
algoritmi uključuju stohastičke elemente: inicijalizacija parametara,
uzorkovanje tijekom treniranja, odabir podskupova podataka. Rezultati
takvih algoritama mogu varirati između pokretanja. Istraživač mora
provjeriti jesu li zaključci robusni na ove varijacije.

Tablica 21 prikazuje dimenzije valjanosti i pouzdanosti u kontekstu
računalne analize teksta.

\begin{longtable}[]{@{}
  >{\raggedright\arraybackslash}p{(\linewidth - 4\tabcolsep) * \real{0.1860}}
  >{\raggedright\arraybackslash}p{(\linewidth - 4\tabcolsep) * \real{0.2093}}
  >{\raggedright\arraybackslash}p{(\linewidth - 4\tabcolsep) * \real{0.6047}}@{}}
\caption{Dimenzije valjanosti i pouzdanosti u računalnoj analizi
teksta}\tabularnewline
\toprule\noalign{}
\begin{minipage}[b]{\linewidth}\raggedright
Aspekt
\end{minipage} & \begin{minipage}[b]{\linewidth}\raggedright
Pitanje
\end{minipage} & \begin{minipage}[b]{\linewidth}\raggedright
Primjer u analizi teksta
\end{minipage} \\
\midrule\noalign{}
\endfirsthead
\toprule\noalign{}
\begin{minipage}[b]{\linewidth}\raggedright
Aspekt
\end{minipage} & \begin{minipage}[b]{\linewidth}\raggedright
Pitanje
\end{minipage} & \begin{minipage}[b]{\linewidth}\raggedright
Primjer u analizi teksta
\end{minipage} \\
\midrule\noalign{}
\endhead
\bottomrule\noalign{}
\endlastfoot
Sadržajna valjanost & Pokriva li mjera sve aspekte koncepta? & Je li
rječnik sentimenta potpun za domenu? \\
Konstruktna valjanost & Odgovara li mjera teorijskom konstruktu? &
Odgovaraju li LDA teme teorijskim konceptima? \\
Prediktivna valjanost & Predviđa li mjera povezane varijable? & Korelira
li automatski sentiment s ljudskim procjenama? \\
Test-retest pouzdanost & Daje li metoda konzistentne rezultate? &
Variraju li rezultati između pokretanja? \\
\end{longtable}

\subsection{Pristranost u podacima}\label{pristranost-u-podacima}

Rezultati računalne analize teksta mogu biti samo onoliko dobri koliko
su dobri podaci na kojima se temelje. Pristranost (bias) može ući u
analizu na više točaka: u konstrukciji korpusa, u označenim podacima za
treniranje i u samim algoritmima.

Pristranost uzorkovanja javlja se kada korpus nije reprezentativan za
populaciju o kojoj želimo zaključivati. Ako analiziramo ``hrvatsko
novinarstvo'', ali naš korpus sadrži samo članke iz dvaju najvećih
portala, naši zaključci neće biti generalizabilni na cjelokupno
novinarstvo. Ako analiziramo ``javno mnijenje'' na temelju objava na
društvenim mrežama, moramo biti svjesni da korisnici društvenih mreža
nisu reprezentativni uzorak populacije: mlađi su, urbaniziraniji,
obrazovaniji.

Temporalna pristranost nastaje kada korpus pokriva samo određeno
vremensko razdoblje. Jezik se mijenja, teme dolaze i odlaze, stilovi
izvještavanja evoluiraju. Model treniran na člancima iz 2015. godine
možda neće dobro funkcionirati na člancima iz 2024. godine. Pandemija
COVID-19 uvela je čitav novi vokabular koji modeli trenirani prije 2020.
neće prepoznati.

Pristranost u označenim podacima posebno je relevantna za nadzirano
učenje. Ljudski koderi koji označavaju podatke za treniranje unose
vlastite pristranosti, nekonzistentnosti i pogreške. Ako su koderi
dominantno jednog političkog uvjerenja, njihove procjene sentimenta ili
kategorizacije mogu sistematski favorizirati jednu stranu.

Algoritmička pristranost odnosi se na sistemske pogreške koje algoritmi
uče iz podataka. Jezični modeli trenirani na velikim korpusima interneta
uče i reproduciraju stereotipe, predrasude i neravnoteže prisutne u tim
podacima. Ova pristranost posebno je problematična jer je često
nevidljiva: model može postizati visoku točnost na standardnim
metrikama, a ipak sistematski diskriminirati određene grupe.

Problem generalizacije i prekomjernog prilagođavanja zaslužuje posebnu
pažnju. Prekomjerno prilagođavanje (overfitting) javlja se kada model
nauči specifičnosti trenažnih podataka umjesto generalizirajućih
obrazaca. Takav model ima izvrsne performanse na trenažnim podacima, ali
loše performanse na novim podacima. U kontekstu analize teksta,
klasifikator može naučiti da su članci određenog autora tipično
pozitivni, pa klasificira kao pozitivno sve što dolazi od tog autora,
čak i kada je sadržaj negativan. Strategije za izbjegavanje prekomjernog
prilagođavanja uključuju podjelu podataka na trenažni, validacijski i
testni skup, unakrsnu validaciju i regularizaciju.

Domenski pomak predstavlja poseban izazov koji nastaje kada se model
primjenjuje na podatke koji dolaze iz drugačije distribucije nego
trenažni podaci. Klasifikator treniran na vijestima iz jednog medija
možda neće funkcionirati na vijestima iz drugog medija koji ima
drugačiji stil, vokabular ili tematski fokus. Čak i unutar istog medija,
članci iz različitih rubrika mogu biti dovoljno različiti da model
treniran na jednoj rubrici loše funkcionira na drugoj. Istraživač mora
pažljivo razmisliti o tome odgovaraju li njegovi podaci za treniranje
podacima na kojima će se model primjenjivati.

\subsection{Interpretabilnost i problem crne
kutije}\label{interpretabilnost-i-problem-crne-kutije}

Različiti algoritmi nude različite stupnjeve interpretabilnosti, odnosno
mogućnosti da istraživač razumije zašto je model donio određenu odluku.

Na jednom kraju spektra nalaze se rječnički pristupi koji su potpuno
transparentni. Ako tekst ima pozitivan sentiment, istraživač može točno
vidjeti koje su riječi tome doprinijele jer se postupak svodi na
prebrojavanje unaprijed definiranih riječi. Ova transparentnost
omogućuje provjeru smislenosti rezultata i identifikaciju potencijalnih
problema.

Logistička regresija i slični linearni modeli također nude relativno
dobru interpretabilnost. Svaka riječ ima koeficijent koji pokazuje
koliko i u kojem smjeru utječe na predikciju. Istraživač može pregledati
riječi s najvišim i najnižim koeficijentima i procijeniti jesu li ti
odnosi smisleni.

Na drugom kraju spektra nalaze se duboki neuronski mrežni modeli poput
transformera koji su notorno teški za interpretaciju. Ovi modeli imaju
stotine milijuna ili čak milijarde parametara organiziranih u kompleksne
arhitekture s višestrukim slojevima nelinearnih transformacija. Zašto je
model određeni tekst klasificirao kao negativan? Odgovor se ne može
jednostavno svesti na nekoliko ključnih riječi jer model uzima u obzir
kompleksne interakcije između riječi, pozicije i konteksta.

Tablica 22 prikazuje stupnjeve interpretabilnosti različitih metoda.

\begin{longtable}[]{@{}
  >{\raggedright\arraybackslash}p{(\linewidth - 6\tabcolsep) * \real{0.1633}}
  >{\raggedright\arraybackslash}p{(\linewidth - 6\tabcolsep) * \real{0.3878}}
  >{\raggedright\arraybackslash}p{(\linewidth - 6\tabcolsep) * \real{0.2245}}
  >{\raggedright\arraybackslash}p{(\linewidth - 6\tabcolsep) * \real{0.2245}}@{}}
\caption{Stupnjevi interpretabilnosti različitih metoda}\tabularnewline
\toprule\noalign{}
\begin{minipage}[b]{\linewidth}\raggedright
Metoda
\end{minipage} & \begin{minipage}[b]{\linewidth}\raggedright
Interpretabilnost
\end{minipage} & \begin{minipage}[b]{\linewidth}\raggedright
Prednosti
\end{minipage} & \begin{minipage}[b]{\linewidth}\raggedright
Nedostaci
\end{minipage} \\
\midrule\noalign{}
\endfirsthead
\toprule\noalign{}
\begin{minipage}[b]{\linewidth}\raggedright
Metoda
\end{minipage} & \begin{minipage}[b]{\linewidth}\raggedright
Interpretabilnost
\end{minipage} & \begin{minipage}[b]{\linewidth}\raggedright
Prednosti
\end{minipage} & \begin{minipage}[b]{\linewidth}\raggedright
Nedostaci
\end{minipage} \\
\midrule\noalign{}
\endhead
\bottomrule\noalign{}
\endlastfoot
Rječnički pristupi & Visoka & Potpuna transparentnost & Ograničena
fleksibilnost \\
Logistička regresija & Visoka & Koeficijenti pokazuju doprinos &
Linearnost može biti ograničenje \\
Stabla odlučivanja & Srednja & Vizualizacija pravila & Nestabilnost,
prekomjerno prilagođavanje \\
SVM & Niska do srednja & Dobra generalizacija & Teško interpretirati u
visokim dimenzijama \\
Neuronske mreže & Niska & Visoka prediktivna moć & ``Crna kutija'' \\
Transformeri (BERT) & Vrlo niska & Vrhunske performanse & Gotovo potpuna
neprozirnost \\
\end{longtable}

\subsection{Etička razmatranja}\label{etiux10dka-razmatranja}

Računalna analiza teksta, posebno kada se primjenjuje na ljudsku
komunikaciju, otvara niz etičkih pitanja koja istraživač mora uzeti u
obzir.

Privatnost postaje sve relevantnija tema kako se metode primjenjuju na
sve veće količine osobnih podataka. Objave na društvenim mrežama,
komentari, poruke -- sve to može sadržavati osjetljive osobne
informacije. Čak i kada su ti tekstovi ``javno'' dostupni, to ne znači
da su njihovi autori pristali na to da budu predmet istraživanja.
Agregirana analiza velike količine javnih objava može otkriti
informacije koje pojedinci nisu namjeravali dijeliti: zdravstveno
stanje, političke stavove, seksualne preferencije, financijsku
situaciju.

Pristanak je složeno pitanje u kontekstu analize teksta. Tradicionalni
model informiranog pristanka teško je primjenjiv kada se analiziraju
tisuće ili milijuni tekstova napisanih od različitih autora. Korištenje
javno dostupnih podataka bez eksplicitnog pristanka može biti zakonito,
ali etička opravdanost ovisi o kontekstu: analizirati javne govore
političara drugačije je od analiziranja osobnih objava privatnih osoba.

Potencijalna zloupotreba istraživačkih rezultata zahtijeva anticipaciju
mogućih štetnih primjena. Metode za detekciju sentimenta mogu se
koristiti za praćenje javnog mnijenja u legitimne svrhe, ali i za
nadzor, manipulaciju ili ciljano uznemiravanje. Metode za klasifikaciju
teksta mogu se koristiti za filtriranje spam poruka, ali i za cenzuru
legitimnog govora.

Transparentnost i reproducijalnost etičke su norme znanstvene prakse
koje dobivaju posebnu težinu u kontekstu računalnih metoda. Istraživač
bi trebao jasno dokumentirati svoje metode, podatke i analitičke odluke
na način koji omogućuje drugima da procijene i repliciraju rezultate.

\subsection{Validacija kao nužnost}\label{validacija-kao-nuux17enost}

S obzirom na sva navedena ograničenja, validacija postaje ne samo
preporučljiva, već nužna komponenta svakog ozbiljnog istraživanja koje
koristi računalnu analizu teksta. Validacija nije jednokratni korak na
kraju analize, već kontinuirani proces koji prožima cijeli istraživački
tijek.

Validacija reprezentacije provjerava odgovara li način na koji smo
reprezentirali tekst našem istraživačkom pitanju. Jesu li izbori koje
smo napravili prilikom pripreme podataka opravdani za naš specifični
kontekst? Jesmo li izgubili relevantne informacije?

Validacija modela provjerava proizvodi li model rezultate koji
odgovaraju ljudskim procjenama. Za klasifikacijske zadatke, to uključuje
usporedbu predikcija modela s ``zlatnim standardom'' ljudskog kodiranja.
Za nenadziranu analizu poput tematskog modeliranja, validacija može
uključivati procjenu koherentnosti tema od strane ljudskih ocjenjivača.

Robusnost zaključaka testira jesu li zaključci osjetljivi na analitičke
odluke. Što se događa ako promijenimo prag za uklanjanje rijetkih
riječi? Ako koristimo drugačiji algoritam? Ako podijelimo podatke
drugačije? Zaključci koji se drže robusnim kroz varijacije u analitičkim
odlukama zaslužuju veće povjerenje od onih koji ovise o specifičnim,
proizvoljnim izborima.

Grimmer, Roberts i Stewart zagovaraju pristup ``ljudi u petlji''
(human-in-the-loop) gdje računalne metode služe kao alat za pojačavanje
ljudskih analitičkih sposobnosti, a ne kao njihova zamjena. Računalo
može identificirati obrasce, sortirati dokumente, kvantificirati
fenomene. Ali interpretacija značenja, procjena relevantnosti,
povezivanje s teorijom i donošenje zaključaka ostaju ljudske zadaće.

\section{Zaključak i budući
pravci}\label{zakljuux10dak-i-buduux107i-pravci}

Područje računalne analize teksta prolazi kroz razdoblje iznimno brzih
promjena. Metode koje su prije samo nekoliko godina predstavljale
vrhunac tehnologije danas su zamijenjene novim pristupima koji postižu
impresivne rezultate na zadacima koji su se činili nedostižnima. Veliki
jezični modeli poput GPT-a, Clauda i Llame transformirali su način na
koji računala procesiraju i generiraju tekst. Multimodalna analiza
proširuje fokus s čistog teksta na integraciju slike, zvuka i videa.
Višejezični modeli smanjuju barijere za jezike s manjim resursima poput
hrvatskog.

\subsection{Veliki jezični modeli}\label{veliki-jeziux10dni-modeli}

Razvoj transformer arhitekture pokrenuo je revoluciju u obradi prirodnog
jezika. Transformeri koriste mehanizam samopažnje (self-attention) koji
omogućuje modelu da pri obradi svake riječi uzme u obzir sve ostale
riječi u tekstu, zahvaćajući složene kontekstualne ovisnosti koje su
bile izvan dosega ranijih arhitektura.

Za istraživanje masovne komunikacije, veliki jezični modeli otvaraju
transformativne mogućnosti. Automatska anotacija tekstova postaje
dostupnija jer se modeli mogu koristiti za klasifikaciju bez opsežnog
ručnog označavanja. Zero-shot i few-shot učenje predstavljaju
paradigmatski pomak jer omogućuju klasifikaciju tekstova samo na temelju
opisa kategorija, bez potrebe za stotinama označenih primjera.

Međutim, s velikim mogućnostima dolaze i značajni izazovi. Neprozirnost
velikih modela nadilazi čak i ranija ograničenja interpretabilnosti.
Halucinacije, odnosno samopouzdano generiranje netočnih informacija,
predstavljaju ozbiljan problem za primjene koje zahtijevaju faktičku
točnost.

\subsection{Multimodalnost i
višejezičnost}\label{multimodalnost-i-viux161ejeziux10dnost}

Tradicionalna analiza teksta fokusira se isključivo na verbalni sadržaj,
no medijska komunikacija sve više uključuje multimodalni sadržaj: tekst
popraćen slikama, video materijali s titlovima, infografike koje
kombiniraju vizualne i tekstualne elemente. Napredak u računalnom vidu i
multimodalnim modelima otvara mogućnosti za integrirane analize koje
zahvaćaju ovu kompleksnost.

Za istraživanje masovne komunikacije, multimodalna analiza posebno je
relevantna jer suvremeni medijski sadržaj rijetko je čisto tekstualan.
Vijesti na portalima redovito uključuju fotografije koje uokviruju
priču. Objave na društvenim mrežama kombiniraju tekst, slike, video i
emoji. Političke kampanje koriste vizualne strategije koje nadopunjuju
ili čak proturječe verbalnim porukama. Analiza koja se ograničava na
tekst propušta značajan dio komunikacijske slike.

Modeli poput CLIP-a, Flaminga ili GPT-4V mogu procesirati istovremeno
tekst i slike, omogućujući pitanja poput: kakav je odnos između tona
naslova i emocionalnog naboja popratne fotografije? Kako se vizualno
predstavljanje političara razlikuje od verbalnog opisa? Je li vizualni
sadržaj konzistentan s tekstualnim ili postoji napetost? Ova pitanja
otvaraju nove istraživačke horizonte za komunikologe.

Za hrvatski jezik posebno je obećavajući razvoj višejezičnih modela.
Modeli poput mBERT-a, XLM-RoBERTa ili višejezičnih verzija GPT-a
trenirani su na tekstovima iz stotinu i više jezika istovremeno. Ovi
modeli mogu prenositi znanje naučeno na jezicima s obilnim resursima na
jezike s manje resursa, uključujući hrvatski. Preliminarna istraživanja
pokazuju da višejezični modeli postižu pristojne rezultate na hrvatskim
zadacima čak i bez specifičnog treniranja na hrvatskom tekstu. S finim
podešavanjem na hrvatske podatke, rezultati se značajno poboljšavaju.

Razvoj sintetičkih podataka i generativne umjetne inteligencije donosi i
nove izazove. Sposobnost velikih jezičnih modela da generiraju
koherentan tekst koji je teško razlikovati od ljudski napisanog teksta
ima dalekosežne implikacije. S jedne strane, sintetički podaci mogu
pomoći istraživačima u proširenju ograničenih trenažnih skupova. S druge
strane, isti kapacitet omogućuje proizvodnju dezinformacija na
industrijskoj skali. Istraživači masovne komunikacije suočavaju se s
rastućim izazovom proučavanja medijskog prostora u kojem značajan dio
sadržaja može biti strojno generiran. Detekcija generiranog teksta
postaje nova istraživačka domena, premda je to utrka u kojoj se napadači
i branitelji neprestano prilagođavaju.

\subsection{Pet principa odgovorne
primjene}\label{pet-principa-odgovorne-primjene}

Ako je nešto jasno iz rasprave o ograničenjima, to je da računalna
analiza teksta zahtijeva promišljenu, kritičku i odgovornu primjenu. Na
temelju literature i iskustva, možemo formulirati pet principa koji bi
trebali voditi istraživače masovne komunikacije u korištenju ovih
metoda.

Prvi princip glasi: budite skromni u tvrdnjama. Rezultati modela su
aproksimacije, ne istine. Nijedan algoritam ne može uhvatiti svu
složenost ljudske komunikacije. Zaključci trebaju odražavati epistemičku
skromnost primjerenu ograničenjima metoda.

Drugi princip zahtijeva: budite transparentni u metodama. Jasno
dokumentirajte sve analitičke odluke, od pripreme podataka do
interpretacije rezultata. Omogućite drugima da procijene i repliciraju
vaš rad. Transparentnost je preduvjet znanstvene kredibilnosti.

Treći princip upozorava: budite skeptični prema rezultatima. Ne
prihvaćajte rezultate zdravo za gotovo samo zato što dolaze iz
sofisticiranog algoritma. Tražite alternativna objašnjenja, testirajte
robusnost, validirajte s ljudskim procjenama.

Četvrti princip naglašava: budite etični u primjeni. Razmislite o
mogućim štetama koje vaše istraživanje može uzrokovati. Poštujte
privatnost subjekata čiji tekstovi se analiziraju. Anticipirajte
potencijalne zloupotrebe vaših metoda i rezultata.

Peti princip savjetuje: budite usmjereni na pitanja, ne na metode.
Metode su alati za odgovaranje na pitanja, ne ciljevi sami po sebi.
Najbolja istraživanja počinju s važnim pitanjima o medijskoj
komunikaciji i društvu, a zatim traže najprikladnije metode za
odgovaranje na ta pitanja.

Tablica 23 sažima pet principa odgovorne primjene.

\begin{longtable}[]{@{}
  >{\raggedright\arraybackslash}p{(\linewidth - 4\tabcolsep) * \real{0.2000}}
  >{\raggedright\arraybackslash}p{(\linewidth - 4\tabcolsep) * \real{0.2889}}
  >{\raggedright\arraybackslash}p{(\linewidth - 4\tabcolsep) * \real{0.5111}}@{}}
\caption{Pet principa odgovorne primjene računalne analize
teksta}\tabularnewline
\toprule\noalign{}
\begin{minipage}[b]{\linewidth}\raggedright
Princip
\end{minipage} & \begin{minipage}[b]{\linewidth}\raggedright
Objašnjenje
\end{minipage} & \begin{minipage}[b]{\linewidth}\raggedright
Praktična implikacija
\end{minipage} \\
\midrule\noalign{}
\endfirsthead
\toprule\noalign{}
\begin{minipage}[b]{\linewidth}\raggedright
Princip
\end{minipage} & \begin{minipage}[b]{\linewidth}\raggedright
Objašnjenje
\end{minipage} & \begin{minipage}[b]{\linewidth}\raggedright
Praktična implikacija
\end{minipage} \\
\midrule\noalign{}
\endhead
\bottomrule\noalign{}
\endlastfoot
Skromnost & Rezultati su aproksimacije & Izbjegavati pretjerane
tvrdnje \\
Transparentnost & Dokumentirati sve odluke & Omogućiti replikaciju \\
Skepticizam & Ne prihvaćati nekritički & Validirati i testirati \\
Etičnost & Anticipirati štete & Poštivati privatnost i dostojanstvo \\
Usmjerenost na pitanja & Metode služe pitanjima & Početi s teorijskim
pitanjima \\
\end{longtable}

\subsection{Završne misli}\label{zavrux161ne-misli}

Računalna analiza teksta nije čarobno rješenje koje automatizira
istraživački proces, već sofisticirani alat koji, kada se koristi
promišljeno, može značajno proširiti doseg i dubinu komunikoloških
istraživanja. Uspješna primjena zahtijeva ne samo tehničke vještine, već
i teorijsku sofisticiranost, kritičku refleksiju i kontinuirani dijalog
između kvantitativnih metoda i kvalitativnog razumijevanja.

S principima odgovorne primjene na umu, budućnost računalne analize
teksta u istraživanju masovne komunikacije izgleda uzbudljivo i
obećavajuće. Izazovi su značajni, ali tako su i mogućnosti. Istraživači
koji se pripreme za ovu budućnost bit će u poziciji da pruže uvide koji
su i metodološki rigorozni i supstantivno značajni, doprinoseći našem
razumijevanju medijske komunikacije u sve složenijem informacijskom
krajoliku.

Završno, analiza teksta podsjeća nas da tekst nije samo spremnik
informacija koje treba izvući, već aktivni konstrukt koji oblikuje naše
razumijevanje svijeta. Mediji ne samo izvještavaju o stvarnosti; oni je
interpretiraju, uokviruju i konstruiraju. Tehnike predstavljene u ovom
poglavlju pružaju alate za sistematsko istraživanje tih konstruktivnih
procesa na razini koja nadilazi pojedinačne tekstove, omogućujući uvide
u šire diskurzivne prakse koje oblikuju javnu sferu.




\end{document}
